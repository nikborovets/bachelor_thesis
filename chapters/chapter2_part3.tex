\subsection{Модель Трухильо}

Модель Трухильо [4] применяется для расчета изменения дисперсных характеристик потока, а именно изменения скоростей капель, их диаметров и количества при взаимодействии со стенкой. Рассмотрим как меняются параметры потока для всех четырех возможных случаев взаимодействия с поверхностью. Начнем с прилипания капли к поверхности.

\noindent Случай 1. Прилипание капли к поверхности

При прилипании капли к поверхности ее скорость зануляется, диаметр не меняется, а сама капля исключается из потока и не участвует в дальнейшем процессе льдообразования.

\noindent Случай 2. Отскок капли от поверхности и поступление во внешний поток

При отскоке капли от поверхности она поступает во внешний поток и ее параметры (указаны на \textit{рисунке 8}) меняются по модели Трухильо. Скорость меняется согласно соотношениям [1]:
\begin{equation}
	f_{u,t} = \frac{u_{t,s}}{u_{t,0}} = 0.85+0.0025\theta_0
\end{equation}
\begin{equation}
	f_{u,n} = \frac{u_{n,s}}{u_{n,0}} = -\left[0.993-0.0307\left(\frac{\pi}{2}-\theta_0\right)+0.0272\left(\frac{\pi}{2}-\theta_0\right)^2-0.0086\left(\frac{\pi}{2}-\theta_0\right)^3\right],
\end{equation}
\begin{equation*}
	\theta_0 = \arctan\left(\frac{u_{t,0}}{u_{n,0}}\right)
\end{equation*}
где $u_{t,s}$ --- тангенциальная составляющая отскочившей капли, $u_{t,0}$ --- тангенциальная составляющая падающей капли, $\theta_0$ --- угол между нормалью к поверхностью и вектором падающей капли, $u_{n,s}$ --- нормальная составляющая отскочившей капли, $u_{n,0}$ --- нормальная составляющая падающей капли. Размер капли при отскоке не меняется. 

\noindent Случай 3. Растекание капли по поверхости.

При растекании по поверхности капля исключается из потока и изменение ее параметров аналогично случаю прилипания.

\noindent Случай 4. Разбрызгивание капли.

Средняя скорость вторичных капель определяется с использованием эмпирической модели Трухильо следующим образом [1]:
\begin{equation}
	\frac{u_{t,s}}{u_{t,0}} = 0.85+0.0025\theta_0
\end{equation}
\begin{equation}
	\frac{u_{n,s}}{u_{n,0}} = -(0.12+0.002\theta_0)
\end{equation}

\begin{figure}[H]
	\centering
	\includegraphics[width=\textwidth]{figures/chill-guy.jpeg}
	\caption*{Рисунок~8 – Соответствующие параметры модели [1].}
	\label{fig:-6}
\end{figure}

При разбрызгивании из одной капли получается множество вторичных капель, количество которых определяется соотношением [4]:
\begin{equation}
	N = \frac{1}{22}\left\{0.0437\left[K\left(\frac{\left|u_d\right|}{u_{d,n}}\right)^2-K_{crit}\right]-44.92\right\}
\end{equation}  
где $K$ --- параметр разбрызгивания Ярина и Вейсса, который считается следующим образом [4]:
\begin{equation}
	K = \frac{3}{2}\left(\frac{LWC}{\rho_d}\right)^{-\frac{3}{8}}\left(Oh^{-\frac{2}{5}}We_n\right)^\frac{5}{16}
\end{equation}  
в который входит $LWC$ --- содержание жидкой воды в атмосфере [4], которое измеряется в граммах на метр кубический, $Oh$ --- критерий подобия Онезорге, характеризующий отношение сил вязкости к силам поверхностного натяжения и инерции [4]:
\begin{equation}
	Oh = \frac{\mu_a}{\sqrt{\rho_a \sigma_d d}}
\end{equation}

Параметр разбрызгивания увеличивается с увеличением диаметра, что делает разбрызгивание более вероятным для крупных переохлажденных капель, чем для обычных капель. При превышении величины параметра критического значения происходит разбрызгивание капли. Величина критического значения параметра Ярина и Вейсса [4]:
\begin{equation}
	K_{crit} = 17
\end{equation}

Учет массы капель, разбрызгивающихся при ударе о поверхность аэродинамического профиля, выполняется с применением эмпирической модели разбрызгивания Трухильо, которая использует понятие коэффициента потери массы $\varphi$. Уравнение было откалибровано для условий крупных переохдажденных капель. Количество массы разбрызганной воды получено из измерений Ярина и Вейсса и приведено в виде отношения объемов вторичных капель к падающим каплям [4]: 
\begin{equation}
	\varphi(K) = \frac{m_s}{m_0} = \frac{3.8}{\sqrt{K}}(1-\exp\{-0.85(K-K_{crit})\})
\end{equation}
где $m_s$ --- масса разбрызганной воды, $m_0$ --- масса падающей воды. Для $\varphi(K)$ зависимость от $K$ показана на \textit{рисунке 9} [4].
\begin{figure}[H]
	\centering
	\includegraphics[width=\textwidth]{figures/chill-guy.jpeg}
	\caption*{Рисунок~9 – Коэффициент потери массы.}
	\label{fig:-7}
\end{figure}
Зная количество вторичных капель и величину потери массы, средний диаметр вторичных капель может быть рассчитан по формуле [4]:
\begin{equation}
	d_s = \left(\frac{\varphi}{N}\right)^{\frac{1}{3}}d
\end{equation}

\subsection{Модель взаимодействия кристаллов со стенкой}

Модель взаимодействия кристалла с твёрдой поверхностью основана на использовании безразмерного числа Видора, которое показывает сооотношение между внутренней энергией кристалла при воздействии на поверхность и его поверхностной энергией [3]:
\begin{equation}
	\mathcal{L}=\frac{\rho_p d_p u^2_{p,n}}{12 e_{\sigma}},
\end{equation}
где $\rho_p$ и $d_p$ --- плотность и диаметр кристалла соответственно, $e_{\sigma}$ --- поверхностная энергия кристалла. Для неё справедлива формула:
\begin{equation}
	e_\sigma=e_{\sigma_0} \exp\left[\frac{Q_s}{R}\left(\frac{1}{T_p}-\frac{1}{T_0}\right)\right],
\end{equation}
где $e_{\sigma_0} =0{,}12\, \text{Дж}/\text{м}^2$ --- начальная поверхностная энергия кристалла, $Q_s = 48{,}2\, \text{кДж}/\text{моль}$ --- энергия активации, $R=8{,}31\,\text{Дж}/(\text{моль} \cdot \text{К})$ --- универсальная газовая постоянная, $T_p$ --- температура кристалла, а $T_0 = 253\, \text{К}$.

На основании различных диапазонов чисел Видора выделяют три основных режима взаимодействия кристалла с поверхностью [3], они приведены в \textit{таблице 2}:

\begin{table}[H]
	\caption*{Таблица 2 - Режимы взаимодействия кристалла с поверхностью}
	\small
	\centering
	\begin{tabular}{|l|c|c|c|}
		\hline
		\makecell{\textbf{Режимы}\\\textbf{взаимодействия}} &
		\makecell{\textbf{Характерные}\\\textbf{числа Видора}} &
		\makecell{\textbf{Результирующая}\\\textbf{скорость}} &
		\makecell{\textbf{Изменение диаметра и}\\\textbf{количества кристаллов}} \\
		\hline
		\makecell{Упругий отскок \\\ кристалла} & $\mathcal{L} \leq 0{,}5$ & \makecell{Меняется как для \\\ упругого отскока} & Не меняется \\
		\hline
		\makecell{Непругий отскок \\\ кристалла} & $0{,}5 < \mathcal{L} \leq 90$ & \makecell{Расчет по модели \\\ MUSIC-haic [3]} & Не меняется \\
		\hline
		\makecell{Фрагментация \\\ кристалла} & $\mathcal{L} > 90$ & \makecell{Расчет по модели \\\ MUSIC-haic [3]} & \makecell{Расчет по модели \\\ MUSIC-haic [3]} \\
		\hline
	\end{tabular}
\end{table}
