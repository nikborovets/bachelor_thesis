\section{Выбор и описание моделей}
\label{sec:model_selection}

\hspace*{1.25cm}На основе выводов, сделанных в Главе 2, для решения задачи прогнозирования были выбраны модели, представляющие два разных подхода: классическую статистику и современное машинное обучение.

\subsection{Модель SARIMA}
\hspace*{1.25cm}Сезонная авторегрессионная интегрированная скользящая средняя (SARIMA) — это статистическая модель, которая является расширением модели ARIMA и предназначена для работы с временными рядами, обладающими ярко выраженной сезонностью [7]. Выбор этой модели обоснован анализом ACF/PACF, который указал на наличие тренда, авторегрессионной зависимости и сезонных колебаний.

\subsection{Модель CatBoost}
\hspace*{1.25cm}CatBoost — это высокопроизводительная реализация градиентного бустинга над деревьями решений [8]. Она хорошо зарекомендовала себя в работе с разнородными табличными данными, эффективно обрабатывает категориальные признаки и не требует тщательной настройки гиперпараметров.

\subsection{Модель LSTM}
\hspace*{1.25cm}Сети с долгой краткосрочной памятью (Long Short-Term Memory, LSTM) — это разновидность рекуррентных нейронных сетей (RNN), специально разработанная для улавливания долгосрочных зависимостей в последовательных данных [9].Также архитектура LSTM позволяет эффективно бороться с проблемой затухающих градиентов.

\hspace*{1.25cm}В данной работе используется LSTM-архитектура со следующими характеристиками:
\begin{itemize}
	\item Входной слой принимает последовательности длиной \texttt{window\_size} временных шагов с количеством признаков, определяемым этапом формирования признаков;
	\item Один LSTM-слой с 64 нейронами;
	\item Полносвязный скрытый слой с 8 нейронами и функцией активации ReLU;
	\item Выходной слой с одним нейроном и линейной функцией активации для регрессии;
	\item Оптимизатор Adam с learning rate 0.0001, функция потерь --- Mean Squared Error.
\end{itemize}