\chapter{Результаты расчета по программе}

В этой главе приводятся итоги моделирования взаимодействия жидких капель и твёрдых кристаллов со стенкой при специально подобранных начальных условиях. Цель — продемонстрировать четыре возможных сценария взаимодействия для капель и три — для кристаллов. Начнем с выбора начальных условий для капель. Все четыре капли моделируются из начальной точки $(0,0)$ с разными скоростями, диаметрами, силами, действующими на них вертикально вниз и скоростями воздуха, которые меняются по линейному закону. Стенка находится на расстоянии $10$ м от старта капель, с параметрами приведенными в \textit{Таблице 3}.

\begin{table}[H]
	\caption*{Таблица 3 — Начальные условия моделирования и результат взаимодействия капель со стенкой}
	\small
	\centering
	\begin{tabular}{|c|c|c|c|c|l|}
		\hline
		\makecell{\textbf{№}\\\textbf{режима}} 
		& \makecell{\textbf{Начальная}\\\textbf{скорость, м/с}} 
		& \makecell{\textbf{Радиус,}\\\textbf{мкм}} 
		& \makecell{\textbf{Скорость}\\\textbf{воздуха, м/с}} 
		& \makecell{\textbf{Внешняя}\\\textbf{сила}} 
		& \makecell{\textbf{Результат}\\\textbf{взаимодействия}} \\
		\hline
		1 & 10  &   20 & 20--50   &  50 & Прилипание       \\ \hline
		2 & 100 &   30 & 120--150 &  50 & Растекание       \\ \hline
		3 & 50  &   30 & 120--150 &  60 & Отскок           \\ \hline
		4 & 120 & 1000 & 120--150 & 150 & Разбрызгивание   \\ \hline
	\end{tabular}
\end{table}

Для кристаллов ситуация аналогична каплям, только для них есть всего три различных варианта взаимодействия с поверхностью. Для кристаллов стенка находится на расстоянии 5 м от их старта из нулевой точки и параметры приведены в \textit{Таблице 4}.

\begin{table}[H]
	\caption*{Таблица 4 — Начальные условия моделирования и результат взаимодействия кристаллов со стенкой}
	\small
	\centering
	\begin{tabular}{|c|c|c|c|c|l|}
		\hline
		\makecell{\textbf{№}\\\textbf{режима}} 
		& \makecell{\textbf{Начальная}\\\textbf{скорость, м/с}} 
		& \makecell{\textbf{Радиус,}\\\textbf{мкм}} 
		& \makecell{\textbf{Скорость}\\\textbf{воздуха, м/с}} 
		& \makecell{\textbf{Внешняя}\\\textbf{сила}} 
		& \makecell{\textbf{Результат}\\\textbf{взаимодействия}} \\
		\hline
		1 & 20  &   50 & 40--60   &  50 & Упругий отскок       \\ \hline
		2 & 50 & 100 & 120--150 &  100 & Неупругий отскок       \\ \hline
		3 & 200  & 1000 & 300--400 &  1000 & Фрагментация           \\ \hline
	\end{tabular}
\end{table}

Изменение диаметра капель приведено на \textit{рисунке 11}. Как видно из рисунка для режимов прилипания, растекания и отскока размер капель не меняется на протяжении всего полета. Для режима фрагментации можно увидеть заметное изменение диаметра при взаимодействии со стенкой и в дальнейшем разрушение в потоке (режим 4 на \textit{рисунке 11}). Согласно соотношению (2.24) получилось шесть вторичных капель.

 \begin{figure}[H]
	\centering
	\includegraphics[width=\textwidth]{figures/chill-guy.jpeg}
	\caption*{Рисунок~11 – Результаты моделирования изменения диаметров капель при взаимодействии со стенкой.}
	\label{fig:-9}
\end{figure}

Если посмотреть на поведение диаметров кристаллов, то оно приведено на \textit{рисунке 12}. Как видно из рисунка диаметры кристаллов не меняются в процессе полета, а меняется только у кристалла в режиме фрагментации (3 режим на \textit{рисунке 12}).

\begin{figure}[H]
	\centering
	\includegraphics[width=\textwidth]{figures/chill-guy.jpeg}
	\caption*{Рисунок~12 – Результаты моделирования изменения диаметров кристаллов при взаимодействии со стенкой.}
	\label{fig:-10}
\end{figure}

Теперь рассмотрим результаты моделирования движения этих капель и кристаллов и изменения их скоростей. Для капель траектория движения показана на \textit{рисунке 13}, а изменение скоростей на \textit{рисунке 14}.

\begin{figure}[H]
	\centering
	\includegraphics[width=\textwidth]{figures/chill-guy.jpeg}
	\caption*{Рисунок~13 – Траектории капель при взаимодействии со стенкой.}
	\label{fig:-11}
\end{figure}

\begin{figure}[H]
	\centering
	\includegraphics[width=\textwidth]{figures/chill-guy.jpeg}
	\caption*{Рисунок~14 – Изменение скоростей капель при взаимодействии со стенкой.}
	\label{fig:-12}
\end{figure}

Для первых двух режимов прилипания и растекания скорость зануляется при взаимодействии со стенкой, а для отскока и фрагментации наблюдается отскок с уменьшением модуля скорости из-за воздуха идущего слева направо и дальнейший разгон по потоку. Рассмотрим теперь поведение траекторий кристаллов (\textit{Рисунок 15}) и изменение их скоростей (\textit{Рисунок 16}).

\begin{figure}[H]
	\centering
	\includegraphics[width=\textwidth]{figures/chill-guy.jpeg}
	\caption*{Рисунок~15 – Траектории кристаллов при взаимодействии со стенкой.}
	\label{fig:-13}
\end{figure}

\begin{figure}[H]
	\centering
	\includegraphics[width=\textwidth]{figures/chill-guy.jpeg}
	\caption*{Рисунок~16 – Изменение скоростей кристаллов при взаимодействии со стенкой.}
	\label{fig:-14}
\end{figure}

Для кристалла в режиме 3 происходит фрагментация на несколько кристаллов с меньшими диаметрами, для которых можно построить распределение согласно формуле (2.47). Максимальный диаметр фрагментированного кристалла получился равен 786 мкм при начальном диаметре 2000 мкм. Распределение показано на \textit{Рисунке 17}.

\begin{figure}[H]
	\centering
	\includegraphics[width=\textwidth]{figures/chill-guy.jpeg}
	\caption*{Рисунок~17 – Распределение фрагментированных кристаллов по диаметрам.}
	\label{fig:-15}
\end{figure} 