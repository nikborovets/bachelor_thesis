\subsection{Модель MUSIC-haic}

Взаимодействие кристалла льда со стенкой представляет собой сложный процесс. Основными параметрами, влияющими на результаты и динамику, являются размер и скорость. Рассмотрим все три возможных варианта поподробнее.

\noindent Случай 1. Упругий отскок кристалла

При упругом отскоке кристалла от стенки его скорость меняется как для упругого оскока, то есть нормальная составляющая к поверхности меняет направление, не меняя величины, а тангенциальная составляющая сохраняется. Размер кристалла при этом не меняется.

\noindent Случай 2. Непругий отскок кристалла

Для неупругого отскока кристалла так же как и для упругого размер не меняется. Скорость изменяется по модели MUSIC-haic согласно соотношению [3]:
\begin{equation}
	\mathbf{v}_{p,s}=\xi_t (\mathbf{v}_p - \mathbf{v}_{p,n})-\xi_{nn} \mathbf{v}_{p,n},
\end{equation}
где $\mathbf{v}_{p,s}$ --- вектор скорости кристалла после отскока от поверхности, $\mathbf{v}_p$ --- вектор скорости перед ударом, $\mathbf{v}_{p,n}$ --- вектор нормальной к стенке скорости перед ударом, $\xi_{t} = 1$ --- коэффициент востановления тангенциальной компоненты скорости, $\xi_{nn}$ --- коэффициент востановления нормальной компоненты скорости, который вычисляется с помощью соотношения [3]:

\begin{equation}
	\xi_{nn} = \left(\frac{\mathcal{L}_1}{\mathcal{L}}\right)^{\frac{1}{3}},  \mathcal{L}_1 = 0.5
\end{equation}

\noindent Случай 3. Фрагментация кристалла 

Средняя скорость фрагментированных кристаллов определяется с помощью модели MUSIC-haic следующим образом [3]:

\begin{equation}
	\mathbf{v}_{p,s}=\xi_t (\mathbf{v}_p - \mathbf{v}_{p,n})-v_{p,n} (\xi_{nt} \mathbf{t}+\xi_{nn} \mathbf{n}),
\end{equation}
где $\mathbf{t}$ и $\mathbf{n}$ --- тангенциальный и нормальный к поверхности единичные вектора соответственно, $\xi_{nt}$ --- коэффициент восстановления, описывающий тангенциальную составляющую скорости, вызванную нормальной входящей составляющей скорости. Он вычисляется по формуле:

\begin{equation}
	\xi_{nt} = k \frac{1}{v_{p,n}} \int\limits_{r_{p,min}}^{r_{p,max}} \frac{\gamma-1}{r_{p,max}^{\gamma-1}-r_{p,min}^{\gamma-1}} r_f^{-\gamma} \int\limits_{0}^{r_p-r_f} \int\limits_{0}^{2 \pi} \int\limits_{0}^{\pi} \frac{1}{\frac{4}{3}(r_p-r_f)^3} k \dot{\varepsilon} (r_p+r\cos \theta) r^2 \sin \theta dr d\theta d\phi dr_f
\end{equation}
после взятия интеграла получается:
\begin{equation}
	\xi_{nt} = k \frac{3}{32}\pi v_{p,n}^{-\frac{1}{2}} \rho_p^{-\frac{1}{4}} Y_c^{\frac{1}{4}}\left[1-\frac{1}{r_p}\frac{r_{p,max}^{2-\gamma}-r_{p,min}^{2-\gamma}}{r_{p,max}^{1-\gamma}-r_{p,min}^{1-\gamma}}\right],
\end{equation}
где параметр $k \approx 0{,}909$ используется для приведения модели в соответствие с имеющимися экспериментальными данными, $\rho_p$ --- плотность льда, $r_p$ --- радиус налетающего кристалла, а $r_{p,max}$ и $r_{p,min}$ --- радиусы самого маленького и самого большого фрагментированного кристалла соответственно, $\gamma = 2{,}73$. Что касается предела текучести при сжатии $Y_c$, то экспериментальные данные
свидетельствуют о том, что он зависит от скорости деформации и от температуры. В качестве зависимости от скорости деформации используется полиномиальная подгонка. Температурная зависимость
квазистатического предела текучести льда при сжатии описывается
экспериментальными данными для которых простая
линейная подгонка дает удовлетворительную точность. Полученная формула гласит [8]:
\begin{equation}
	Y_c(T,\dot{\varepsilon})=Y_{c,0} f_{y_c}(\dot{\varepsilon}) g_{y_c}(T),
\end{equation}
где контрольное значение квазистатического предела текучести при сжатии $Y_{c,0} = 5{,}2 \cdot 10^{6}\,\text{Па}$, а для $f_{y_c}(\dot{\varepsilon})$ и $g_{y_c}(T)$ справедливы соотношения [8]:
\begin{equation}
	\begin{split}
		f_{y_c}(\dot{\varepsilon}) = \exp\bigg(
		&3.596 \cdot 10^{-4} \ln^3 \left(\frac{\dot{\varepsilon}}{\dot{\varepsilon}_0}\right) 
		-7.47 \cdot 10^{-3} \ln^2 \left(\frac{\dot{\varepsilon}}{\dot{\varepsilon}_0}\right) \\
		&+1.118 \cdot 10^{-1} \ln \left(\frac{\dot{\varepsilon}}{\dot{\varepsilon}_0}\right)
		+1.062 \bigg)
	\end{split}
\end{equation}
\begin{equation}
	g_{y_c}(T)=-28.7\left(\frac{T}{T_{c,0}}\right)+30.21,
\end{equation}
где $\dot{\varepsilon}$ --- скорость деформации кристалла, а $\dot{\varepsilon}_0$ --- параметр нормализации скорости деформации, $T_{c,0}=273\, \text{К}$. Оценка скорости деформации [8]: 
\begin{equation}
	\dot{\varepsilon} \sim v_{n,0}^{\frac{1}{2}} d_p^{-1} \rho_p^{-\frac{1}{4}} Y_c^{\frac{1}{4}}
\end{equation}

Как видно из уравнения (2.40) для того чтобы посчитать $\dot{\varepsilon}$ необходимо знать $Y_c$, но для $Y_c$ в свою очередь нужно знание $\dot{\varepsilon}$. Проблема решается просто: сначала для расчета $\dot{\varepsilon}$ по формуле (2.40) вместо $Y_c$ подставляется $Y_{c,0}$, а затем уже по формуле (2.37) вычисляется $Y_c$. Таким образом происходит итеративная процедура до тех пор пока не достигнется желаемая точность. Для $\dot{\varepsilon}_0$ справедливо соотношение [8]:
\begin{equation}
	\dot{\varepsilon}_0 = \frac{1}{6} P_{s,0}^3 c^{-3} \rho_p^{-2} G_c^{-1},
\end{equation}
где $P_{s,0} = 10^6\, \text{Па}$ --- постоянное исходное растягивающее давление, $c$ --- скорость звука и $G_c$ --- энергия на единицу площади, необходимая для создания новой поверхности разлома. Скорость звука в кристалле льда вычисляется как:
\begin{equation}
	c = \sqrt{\frac{E}{\rho_p}},
\end{equation}
где $E = 10{,}5 \cdot 10^9\,\text{Па}$ --- модуль Юнга.
\begin{equation}
	G_c = G_{c,0} \exp \left(\frac{Q_s}{R T}-\frac{Q_s}{R T_0}\right),
\end{equation}
где $G_{c,0} = 0{,}12\,\text{Дж}/\text{м}^2$.

Рассмотрим дальше какие получаются диаметры фрагментированных кристаллов. Для этого сначала считается максимальный диаметр фрагментированного кристалла [3]:
\begin{equation}
	d_{p,max}=C_f s_0 \left(\frac{\dot{\varepsilon}}{\dot{\varepsilon}_0}\right)^\alpha,
\end{equation}
где $C_f= 42{,}824$, $\alpha=-0{,}885$ [8], $s_0$ --- параметр нормализации для размера фрагмента, который вычисляется согласно соотношению [8]:
\begin{equation}
	s_0=12 E G_c P_{s,0}^{-2}
\end{equation}
Затем считается минимальный диаметр фрагментированного кристалла:
\begin{equation}
	d_{p,min}=0.015 d_{p,max}
\end{equation}
Теперь зная размер самого маленького и самого больших фрагментов льда строится распределение кристаллов по размерам [9]:
\begin{equation}
	Q(s)=1-\exp\left[-\left(\frac{s}{s_c}\right)^{k_s}\right],
\end{equation}
где $s_c=0{,}75$, $k_s=5{,}8$, $s = d/d_{p,max}$. $Q(s)$ рассчитывается путем суммирования массы
всех фрагментов, размеры которых меньше заданного размера $s$. Размер
фрагмента оценивается как диаметр сферы с одинаковой
массой, значения нормируются на диаметр образца. На \textit{рисунке 10} [9] можно видеть экспериментальные распределения по размерам, полученные для разных скоростей соударения кристаллов со стенкой. Зеленой пунктирной линией показана аппроксимация формулой (2.47). При увеличении скорости соударения кривая распределения смещается влево, то есть уменьшается максимальный диаметр получившихся осколков.

\begin{figure}[H]
	\centering
	\includegraphics[width=\textwidth]{figures/chill-guy.jpeg}
	\caption*{Рисунок~10 – Распределение фрагментов по размерам.}
	\label{fig:-8}
\end{figure} 