\section{Методология проведения экспериментов}
\label{sec:experiment_methodology}

Корректная оценка качества моделей временных рядов требует специального подхода к разделению данных, учитывающего временную структуру и предотвращающего утечку информации из будущего в прошлое.

\subsection{Кросс-валидация для временных рядов}

Для корректной оценки качества моделей применяется специализированная кросс-валидация временных рядов (TimeSeriesSplit), которая учитывает хронологический порядок данных и предотвращает утечку информации из будущего.

Метод TimeSeriesSplit работает следующим образом:
\begin{itemize}
	\item Данные разбиваются на $k$ фолдов, где каждый последующий фолд включает больше исторических данных для обучения;
	\item Для каждого фолда тестовая выборка всегда находится хронологически после обучающей;
	\item Внутри каждого фолда обучающие данные дополнительно разделяются на train и validation в пропорции, определяемой параметром \texttt{test\_size}.
\end{itemize}

\subsection{Процедура валидации}

Для каждого фолда кросс-валидации выполняется следующая последовательность действий:
\begin{enumerate}
	\item \textbf{Разделение данных}: фолд разбивается на train+validation и test согласно TimeSeriesSplit;
	\item \textbf{Внутреннее разделение}: train+validation дополнительно разделяется на обучающую и валидационную выборки;
	\item \textbf{Масштабирование}: параметры нормализации вычисляются только на обучающей выборке и применяются ко всем частям фолда;
	\item \textbf{Обучение модели}: модель обучается на train с валидацией на validation выборке;
	\item \textbf{Оценка качества}: финальная оценка производится на тестовой части фолда;
	\item \textbf{Сохранение результатов}: метрики каждого фолда сохраняются для последующего усреднения.
\end{enumerate}

Итоговые метрики качества вычисляются как среднее арифметическое соответствующих метрик по всем фолдам, что обеспечивает более надежную и несмещенную оценку производительности модели.

\subsection{Горизонт прогнозирования}

Все модели настраиваются для прогнозирования на 900 временных шагов вперед (3.75 часа), что соответствует практическим требованиям системы мониторинга для своевременного реагирования на потенциальные проблемы в видеоконвейере. 