\section{Модель разрушения капель в потоке}

Капли могут испытывать колебательные деформации, которые приводят к разрушению, вызванному ускорением, под действием аэродинамических сил. Изменение диаметра капли в процессе её разрушения определяется следующим дифференциальным уравнением в частных производных [1]:
\begin{equation}
	\frac{\mathrm{d} d}{\mathrm{d} t} = \frac{d_{stab} - d_0}{T},
\end{equation}
где $d_{stab}$ --- стабильный диаметр капли, $d_0$ --- начальный диаметр капли, $T$ --- безразмерное время разрыва капли. Для капель выделяют пять
различных механизмов разрушения, характеризующихся первоначальным состоянием капли.
Безразмерное время распада $T$ определяется следующим образом [1]:
\begin{equation}
	T = 
	\begin{cases}
		6 (We - 12)^{-0.25}, & 12 \leq We < 18 \\
		2.45(We-12)^{0.25}, & 18 \leq We < 45 \\
		14.1(We-12)^{-0.25}, & 45 \leq We < 350 \\
		0.766(We-12)^{0.25}, & 350 \leq We < 2700 \\
		5500, & We \geq 2700 
	\end{cases}
\end{equation}
Число Вебера, входящее в (2.15) определяется по формуле [1]:
\begin{equation}
	We = \frac{\rho_a \left| \mathbf{V}_a - \mathbf{V}_d \right|^2 d}{\sigma_d},
\end{equation}
где $\sigma_d$ --- коэффициент поверхностного натяжения капли. Максимальный устойчивый диаметр капли при прекращении всех процессов распада определяется с помощью критического числа Вебера [1]:
\begin{equation}
	We_{crit} = 12 \Rightarrow d_{stab} = \frac{12 \sigma_d}{\rho_a \left| \mathbf{V}_a - \mathbf{V}_d \right|^2},
\end{equation}

\section{Модель взаимодействия частиц со стенкой}

\subsection{Модель взаимодействия капель со стенкой}

При полете в канале капли и кристаллы могут взаимодействовать со стенками этого канала. Модель взаимодействия капли с твёрдой стенкой зависит от её скоростных и теплофизических параметров. У капель выделяют четыре различных режима представленных на \textit{рисунке 7} и основанных на безразмерном нормальном к стенке числе Вебера [4]:
\begin{equation}
	We_n = \frac{\rho_a \mathbf{V}_{d,n}^2 d}{\sigma_d},
\end{equation}
где $\mathbf{V}_{d,n}$ --- нормальная к поверхности скорость капли.

\begin{figure}[H]
	\centering
	
	\begin{subfigure}[t]{0.45\textwidth}
		\centering
		\includegraphics[width=\textwidth]{figures/chill-guy.jpeg}
		\caption{Прилипание}
		\label{fig:sub1}
	\end{subfigure}
	\hfill
	\begin{subfigure}[t]{0.45\textwidth}
		\centering
		\includegraphics[width=\textwidth]{figures/chill-guy.jpeg}
		\caption{Отскок}
		\label{fig:sub2}
	\end{subfigure}
	
	\vspace{1em}
	
	\begin{subfigure}[t]{0.45\textwidth}
		\centering
		\includegraphics[width=\textwidth]{figures/chill-guy.jpeg}
		\caption{Растекание}
		\label{fig:sub3}
	\end{subfigure}
	\hfill
	\begin{subfigure}[t]{0.45\textwidth}
		\centering
		\includegraphics[width=\textwidth]{figures/chill-guy.jpeg}
		\caption{Разбрызгивание}
		\label{fig:sub4}
	\end{subfigure}
	
	\caption*{Рисунок~7 - Варианты взаимодействия капли с твёрдой поверхностью}
	\label{fig:4grid}
\end{figure}

На основании различных диапазонов чисел Вебера определены основные режимы взаимодействия капли с поверхностью [4], они приведены в \textit{таблице 1}:

\begin{table}[H]
    \caption*{Таблица 1 - Режимы взаимодействия капли с поверхностью}
    \small
	\centering
	\begin{tabular}{|l|c|c|c|}
		\hline
		\makecell{\textbf{Режимы}\\\textbf{взаимодействия}} &
		\makecell{\textbf{Характерные}\\\textbf{числа Вебера}} &
		\makecell{\textbf{Результирующая}\\\textbf{скорость}} &
		\makecell{\textbf{Изменение диаметра и}\\\textbf{количества капель}} \\
		\hline
		\makecell{Прилипание капли \\\ к поверхности} & $We \leq 2$ & Скорость зануляется & Исключается из потока \\
		\hline
		Отскок капли & $2 < We \leq 10$ & Упругий отскок [4] & Не меняется \\
		\hline
		\makecell{Растекание капли \\\ по поверхности} & $10 < We \leq 1320La^{-0.183}$ & Скорость зануляется & Исключается из потока \\
		\hline
		\makecell{Реализуется \\\ разбрызгивание} & $We > 1320La^{-0.183}$ & \makecell{Вычисляется по \\\ модели Трухильо [4]} & \makecell{Количество и размер \\\ капель вычисляется \\\ по модели Трухильо [4]} \\
		\hline
	\end{tabular}
\end{table}
где число Лапласа вычисляется следующим образом:
\begin{equation}
	La = \frac{\sigma_d \rho_d d}{\mu_d^2},
\end{equation}
$\mu_d$ --- динамическая вязкость воды. Далее рассмотрим модель Трухильо для расчета изменения параметров взаимодействующих капель с поверхностью.
