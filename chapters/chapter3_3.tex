\section{Метрики оценки качества}
\label{sec:evaluation_metrics}

Для оценки качества моделей прогнозирования используется набор метрик, позволяющих комплексно оценить точность оценки задержек. Выбор метрик обусловлен спецификой временных рядов и требованиями к практическому применению системы.

\subsection{Средняя абсолютная процентная ошибка (MAPE)}

MAPE является основной метрикой для оценки качества, поскольку обеспечивает интерпретируемость результатов в процентах:

\begin{equation}
	\text{MAPE} = \frac{100\%}{n} \sum_{i=1}^{n} \left| \frac{y_i - \hat{y}_i}{y_i} \right|
\end{equation}

где $y_i$ — истинное значение, $\hat{y}_i$ — прогнозируемое значение, $n$ — количество наблюдений. Согласно техническим требованиям, целевое значение MAPE должно быть менее 10\%.

\subsection{Среднеквадратичная ошибка (RMSE)}

RMSE чувствительна к выбросам и позволяет оценить общую точность модели:

\begin{equation}
	\text{RMSE} = \sqrt{\frac{1}{n} \sum_{i=1}^{n} (y_i - \hat{y}_i)^2}
\end{equation}

\subsection{Средняя абсолютная ошибка (MAE)}

MAE менее чувствительна к выбросам и показывает среднее отклонение прогнозов:

\begin{equation}
	\text{MAE} = \frac{1}{n} \sum_{i=1}^{n} |y_i - \hat{y}_i|
\end{equation} 