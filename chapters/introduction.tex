\newpage
\chapter*{ВВЕДЕНИЕ}  

\addcontentsline{toc}{chapter}{ВВЕДЕНИЕ}

\textbf{Обоснование выбора темы и актуальность}

Современные системы видеоаналитики играют критически важную роль в обеспечении безопасности и мониторинга объектов критической инфраструктуры, включая аэродромы, железнодорожные станции, морские порты, промышленные предприятия и нефтеперерабатывающие комплексы [1]. Эти системы обрабатывают огромные объемы видеоданных в режиме реального времени, что предъявляет высокие требования к производительности и надежности всего технологического конвейера.

С ростом масштабов развертывания и усложнением архитектуры видеоаналитических систем возрастает и сложность их мониторинга [15]. Современные решения часто включают в себя многоуровневые конвейеры обработки, начиная от захвата видеопотоков с камер, их предварительной обработки, применения алгоритмов машинного обучения для детекции объектов и событий, передачи результатов через брокеры сообщений в бэкенд-системы и далее к конечным пользователям через веб-интерфейсы [12], [13], [16].

Повышение объемов данных и жестких требований к end-to-end задержкам (от момента возникновения события на видео до его отображения оператору) делает необходимым переход от реактивного к предиктивному подходу в управлении производительностью. Традиционные методы мониторинга, основанные на статических пороговых значениях и алертах по факту превышения SLA, не способны предотвратить деградацию качества обслуживания до ее критических проявлений [14].

В данном контексте особую важность приобретает разработка интеллектуальных систем предиктивного анализа, способных на основе потоковых метрик мониторинга (например, собираемых системой Prometheus [2] и визуализируемых в Grafana [3]) заблаговременно оценивать потенциальные проблемы производительности и инициировать превентивные меры по их устранению.

\textbf{Цель и задачи исследования}

\textit{Цель работы:} разработать и внедрить комплексный метод предиктивного анализа задержек в конвейере видеоаналитики, способный прогнозировать конечную метрику $common\_event\_delay$ (также известную как \\$common\_cad$) с заданной точностью для предупреждения операторов о потенциальных сбоях до их фактического проявления.

Достижение поставленной цели требует решения комплекса взаимосвязанных \textit{задач:}

\begin{enumerate}
	\item Проведение аналитического обзора и систематизация современной литературы по предиктивному анализу временных рядов, машинному и глубокому обучению в контексте мониторинга и диагностики производительности систем реального времени [19].
	\item Проведение анализа структуры и взаимных корреляций временных рядов метрик, собираемых на этапах видеоконвейера, включая выявление скрытых зависимостей между компонентами системы и идентификацию наиболее информативных признаков для прогнозирования.
	\item Систематический обзор и сравнительный анализ современных методов прогнозирования временных рядов и обнаружения аномалий [18], включая классические статистические подходы, методы машинного обучения и глубокие нейронные сети, с оценкой их применимости к специфике видеоаналитических конвейеров.
	\item Обоснованный выбор оптимальной архитектуры модели (градиентный бустинг, нейронные сети) с учетом требований к точности и скорости inference, а также определение необходимого объёма обучающих данных и оптимальной периодичности переобучения модели.
	\item Проектирование и реализация полноценного MLOps-конвейера, включающего автоматизированный feature‑engineering, механизмы периодического дообучения модели на новых данных, высокопроизводительный inference‑сервис и системы мониторинга качества оценок.
	\item Всестороннее экспериментальное исследование точности и производительности разработанной модели на обширных исторических данных с использованием методов временной кросс-валидации и оценкой устойчивости к различным типам аномалий в данных.
	\item Разработка и внедрение интеллектуальной системы оповещений с адаптивными порогами, а также формулирование практических рекомендаций по эксплуатации, настройке и масштабированию решения в производственной среде.
\end{enumerate}

\textbf{Методология и методы исследования}

Для достижения поставленной цели и решения сформулированных задач применяется комплексная методология, сочетающая теоретические исследования с практическими экспериментами:

\begin{enumerate}
	\item Организация непрерывного сбора и интеллектуальной предобработки потоковых метрик из системы мониторинга Prometheus [2], включая очистку от выбросов, нормализацию, обработку пропущенных значений и синхронизацию временных рядов различных компонентов системы.
	\item Разработка специализированного модуля построения многомерных временных рядов с интеллектуальной генерацией признаков, включая временные лаги различной глубины, скользящие статистические агрегаты, спектральные характеристики и высокоразмерные эмбеддинги для захвата сложных временных зависимостей.
	\item Реализация и экспериментальное сравнение различных архитектур моделей (градиентный бустинг, нейронные сети) с применением строгих методов перекрёстной валидации по времени для обеспечения корректной оценки обобщающей способности.
	\item Контейнеризация решения с использованием технологии Docker и проведение детальных измерений latency inference в условиях, максимально приближенных к производственным, включая тестирование под нагрузкой и оценку масштабируемости.
\end{enumerate}

\textbf{Теоретическая и практическая значимость}

\textit{Теоретическая значимость} работы заключается в сравнительном анализе методов прогнозирования временных рядов для систем мониторинга и определении их применимости к задачам предиктивной диагностики в условиях жестких временных ограничений.

\textit{Практическая значимость} определяется разработкой готового к промышленному использованию решения для мониторинга и предупреждения отказов видеоконвейера с гарантированным соблюдением SLA по конечной метрике $common\_event\_delay$. Созданная система может быть адаптирована и масштабирована для применения в различных отраслях, где критична надежность систем обработки потоковых данных в реальном времени, включая телекоммуникации, финансовые технологии и промышленную автоматизацию.
