\newpage
\chapter*{ВВЕДЕНИЕ}  

\addcontentsline{toc}{chapter}{ВВЕДЕНИЕ}

\textbf{Тема:}\\
\hspace*{1.25cm}Моделирование трансформации дисперсных характеристик трёхфазного потока в процессе движения по каналу сложной формы.

\textbf{Актуальность}\\
\hspace*{1.25cm}Лёд может образовываться на аэродинамических поверхностях двигателя, особенно при низких температурах и наличии атмосферной воды. Эта вода поступает в виде переохлаждённых капель или кристаллов льда, и накопление льда приводит к ухудшению аэродинамических характеристик, повреждениям деталей (\textit{Рисунок 1}) и даже блокировке потока. Такие явления могут вызвать серьезные аварийные ситуации — например, падение оборотов двигателя или погасание камеры сгорания, повреждение проточной части компрессора.

В связи с этим как в нашей стране, так и за рубежом разработаны новые нормативные документы, касающиеся безопасности полетов, которые определяют два типа обледенения: связанное с наличием в атмосфере жидких переохлаждённых капель, и обледенение, вызванное присутствием в атмосферном облаке ледяных кристаллов или смеси фаз. Особенно важным становится рост летных происшествий, связанных с попаданием в условия ледяных кристаллов. Это существенно влияет на безопасность полётов, поскольку даже незначительное нарушение в работе двигателя может иметь катастрофические последствия.

Данная работа направлена на моделирование трансформации дисперсных характеристик трёхфазного потока (воздух – жидкие капли – ледяные кристаллы) в канале сложной формы, чтобы получить распределение по скоростям и по размерам частиц внутри двигателя. Эти данные необходимы для предсказания процессов обледенения на поверхностях двигателя, лопатках, а также как входные данные для дальнейшего расчета обледенения внутри компрессора.

\begin{figure}[H]
	\centering
	% Две картинки в одном ряду
	\begin{subfigure}{0.25\textwidth}
		\centering
		\includegraphics[width=\textwidth]{figures/chill-guy.jpeg}
		\label{fig:pic1}
	\end{subfigure}
	\hspace{0.7em}
	\begin{subfigure}{0.35\textwidth}
		\centering
		\includegraphics[width=\textwidth]{figures/chill-guy.jpeg}
		\label{fig:pic2}
	\end{subfigure}
	
	\vspace{1em}
	
	% Третья картинка под ними
	\begin{subfigure}{0.6\textwidth}
		\centering
		\includegraphics[width=\textwidth]{figures/chill-guy.jpeg}
		\label{fig:pic3}
	\end{subfigure}
	
	\caption*{Рисунок~1 - Повреждения лопаток компрессора двигателя}
	\label{fig:all_pics}
\end{figure}

\textbf{Цель работы}\\
\hspace*{1.25cm}Основной целью данной работы является разработка программы на которой производится отработка физической и математической модели процесса движения и фрагментации капель и кристаллов в проточной части компрессора и расчёт  с помощью этой программы этих процессов.

\textbf{Задачи}

\vspace{0.5em}

Для достижения поставленной цели, были сформулированы и решены следующие задачи:
\begin{enumerate}[label=\arabic*)\hspace{1em}, leftmargin=2cm, itemsep=0em]
	\item Изучение видов обледенения;
	\item Изучение моделей движения и фрагментации капель и кристаллов в потоке и при ударе о стенку;
	\item Разработка программы, позволяющей рассчитывать процессы движения и фрагментации частиц в процессе их движения по каналу сложной формы;
	\item Расчёт движения частиц, фрагментации, а также их динамических и дисперсных характеристик в процессе взаимодействия с элементами двигателя;
\end{enumerate} 