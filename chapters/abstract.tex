\chapter*{АННОТАЦИЯ}
\addcontentsline{toc}{chapter}{АННОТАЦИЯ}

\hspace*{1.25cm}Выпускная квалификационная работа посвящена разработке метода предиктивного анализа задержек в конвейере видеоаналитики для мониторинга объектов инфраструктуры. \textbf{Цель работы} --- создать алгоритм прогнозирования метрики $common\_event\_delay$ с автоматическим обнаружением аномалий для предупреждения операторов о потенциальных сбоях. В работе применяются \textbf{методы исследования}, включающие анализ временных рядов Prometheus-метрик, сравнение архитектур ML-моделей (таких как LSTM и градиентный бустинг), временную кросс-валидацию, развертывание в Docker и A/B-тестирование. В результате исследования разработан MLOps-конвейер с точностью прогнозирования, превышающей базовые методы, и временем отклика менее 1 секунды. Создана система оповещений с адаптивными порогами. Проведена валидация разработанного решения на исторических данных объемом 90643 точки, собранных за 16 дней. \textbf{Практическая значимость} работы заключается в создании готового к использованию решения для предиктивного мониторинга видеосистем с возможностью адаптации для применения в телекоммуникациях и промышленной автоматизации.

\vspace*{1cm}
\hspace*{1.25cm}\textbf{Ключевые слова:} предиктивный анализ, задержки, видеоаналитика, Prometheus, временные ряды, аномалии.

\newpage 