\chapter*{АННОТАЦИЯ}
\addcontentsline{toc}{chapter}{АННОТАЦИЯ}

\hspace*{1.25cm}Настоящая работа посвящена проблеме обеспечения производительности современных систем видеоаналитики. Для предотвращения деградации качества обслуживания был разработан метод прогнозирования сквозной задержки обработки видеопотока.

\hspace*{1.25cm}В ходе исследования был проведен анализ многомерных временных рядов системных метрик, на основе которого были построены и сравнены прогностические модели, включая CatBoost и LSTM. Для оценки их качества и устойчивости применялась методология временной кросс-валидации.

\hspace*{1.25cm}Результатом работы стал прототип системы предиктивного мониторинга, способный с точностью MAPE 0.89\% прогнозировать задержки на горизонте 3.75 часа. Разработанное решение является основой для создания интеллектуальной системы оповещений, которая позволяет принимать превентивные меры для поддержания стабильности видеоаналитического конвейера.

\vspace*{1cm}
\hspace*{1.25cm}\textbf{Ключевые слова:} предиктивный анализ, временные ряды, мониторинг производительности, видеоаналитика, машинное обучение, LSTM, CatBoost, MLOps.

\newpage 