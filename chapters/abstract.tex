\chapter*{АННОТАЦИЯ}
\addcontentsline{toc}{chapter}{АННОТАЦИЯ}

\hspace*{1.25cm}Выпускная квалификационная работа посвящена разработке метода предиктивного анализа задержек в конвейере видеоаналитики для мониторинга объектов критической инфраструктуры.

\hspace*{1.25cm}\textbf{Цель работы:} создать алгоритм прогнозирования метрики \\ $common\_event\_delay$ с автоматическим обнаружением аномалий для предупреждения операторов о потенциальных сбоях.

\hspace*{1.25cm}\textbf{Методы исследования:} анализ временных рядов Prometheus-метрик, сравнение архитектур ML-моделей (трансформеры, градиентный бустинг), временная кросс-валидация, Docker-развертывание, A/B-тестирование.

\hspace*{1.25cm}\textbf{Основные результаты:} Разработан MLOps-конвейер с точностью прогнозирования, превышающей базовые методы, и временем отклика <1 сек. Создана система оповещений с адаптивными порогами. Проведена валидация на данных 90,644 точки за 16 дней.

\hspace*{1.25cm}\textbf{Практическая значимость:} Готовое решение для предиктивного мониторинга видеосистем критической инфраструктуры с возможностью адаптации для телекоммуникаций и промышленной автоматизации.

\vspace*{1cm}
\hspace*{1.25cm}\textbf{Ключевые слова:} предиктивный анализ, задержки, видеоаналитика, Prometheus, временные ряды, аномалии.

\newpage 