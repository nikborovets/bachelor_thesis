\chapter*{АННОТАЦИЯ}
\addcontentsline{toc}{chapter}{АННОТАЦИЯ}

В настоящей работе разработана и исследована физико-математическая модель трансформации дисперсных характеристик трёхфазного потока, включающего воздух, капли воды и кристаллы льда, при их движении по каналу сложной формы. Актуальность исследования обусловлена задачей прогнозирования процессов обледенения газотурбинных двигателей, при расчете которых необходима информация о динамических и дисперсных характеристиках потока.

В основу моделирования положены безразмерные параметры (число Вебера, Лапласа, Рейнольдса, параметр разбрызгивания Ярина—Вейсса и число Видора), позволяющие выделить четыре режима взаимодействия капель с твёрдой поверхностью — прилипание, отскок, растекание и разбрызгивание, а также три режима поведения ледяных кристаллов — упругий и неупругий отскок, фрагментация. Для частиц сформулированы дифференциальные уравнения движения и их фрагментации при взаимодействии со стенками канала.

Результаты численного моделирования демонстрируют зависимость траекторий и изменения диаметров частиц от начальных скоростей, размеров и условий потока. Показано, что для капель малого радиуса преобладают режимы прилипание и растекание, тогда как при увеличении числа Вебера наблюдается фрагментация и разбрызгивание. Аналогичные результаты получены для кристаллов льда.

Предложенные модели и полученные зависимости могут быть использованы при оптимизации противообледенительных систем, так как могут быть полезны для предсказания входных параметров потока в компрессор, что важно для его дальнейшего расчета обледенения.

\newpage 