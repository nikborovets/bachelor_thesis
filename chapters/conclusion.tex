\chapter*{ЗАКЛЮЧЕНИЕ}
\addcontentsline{toc}{chapter}{ЗАКЛЮЧЕНИЕ}

\noindent 1. Разработка и верификация моделей

В ходе проделанной работы были созданы два программных кода — для жидкой капли и для твёрдого кристалла — в которых реализованы физические алгоритмы движения в потоке, с учётом вязкости, внешних сил и взаимодействия со стенкой. Для капли учтены капиллярные и инерционные силы (числа Вебера и Рейнольдса), для кристалла — упругие, пластические и хрупкие механизмы (число Видора, модуль Юнга, энергия поверхности). Обе программы корректно отражают процессы, протекающие в трехфазном потоке при его движении по каналу сложной формы.

\noindent 2. Демонстрация всех сценариев взаимодействия

Для капель показаны четыре чётко различимых поведения при ударе о поверхность:

 \begin{enumerate}
	\item прилипание,
	\item растекание,
	\item неупругий отскок,
	\item разбрызгивание.
\end{enumerate}
В дополнение к этому показано изменение диаметра капли в процессе полета из-за действия аэродинамических сил.

Для кристаллов продемонстрировано три режима:

\begin{enumerate}
	\item упругий отскок,
	\item неупругий отскок,
	\item разрушение (фрагментация).
\end{enumerate}

Переходы между этими режимами управляются безразмерными числами ($We$ для капли и $\mathcal{L}$ для кристалла), что подтверждено серией численных экспериментов.

\noindent 3. Анализ влияния параметров

Увеличение начальной скорости или размера частицы ведёт к росту безразмерных чисел и переходу к более жёстким режимам (прилипание $\rightarrow$ отскок $\rightarrow$ фрагментация).

Линейный профиль скорости воздуха и наличие внешней силы позволяют моделировать различные условия (падение под углом, сдвиг из-за силы).

\noindent 4. Перспективы развития

\begin{enumerate}
	\item Расширение моделей за счёт учёта фазовых переходов частиц в процессе движеия;
	\item Изменение подхода с моделирования каждой отдельной частицы на моделирование поля течения в целом, то есть переход от лагранжева подхода к эйлеровому;
	\item Добавление более сложных законов изменения скорости воздуха с учетом модели описанной в \textit{разделе 2.1}.
\end{enumerate}

В целом проделанная работа продемонстрировала, что предложенный подход сочетает простоту реализации и достаточную физическую достоверность, позволяя исследовать широкий диапазон сценариев взаимодействия частиц с твёрдыми поверхностями. Полученные результаты корректно отражают процессы, протекающие в трехфазном потоке при его движении по каналу сложной формы. 