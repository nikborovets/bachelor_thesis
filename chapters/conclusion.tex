\chapter*{ЗАКЛЮЧЕНИЕ}
\addcontentsline{toc}{chapter}{ЗАКЛЮЧЕНИЕ}

\hspace*{1.25cm}В настоящей выпускной квалификационной работе было проведено комплексное исследование, направленное на разработку и внедрение метода предиктивного анализа задержек в конвейере видеоаналитики. В ходе работы были решены следующие задачи, позволившие достичь поставленной цели.

\begin{enumerate}
    \item \textbf{Проведен аналитический обзор литературы}, в рамках которого были систематизированы подходы к анализу временных рядов в контексте мониторинга систем реального времени. Это позволило сформировать теоретическую базу для дальнейшего исследования и определить круг потенциально применимых методов.

    \item \textbf{Выполнен всесторонний анализ данных}, собранных с промышленной системы видеоаналитики. Была выявлена суточная сезонность и нелинейный тренд в целевой метрике, а также определены наиболее коррелирующие с ней признаки, в частности, задержка в брокере сообщений Kafka. Результаты этого этапа легли в основу формирования признакового пространства для моделей.

    \item \textbf{Проведен сравнительный анализ моделей}, включающий классическую статистическую модель SARIMA, модель градиентного бустинга CatBoost и рекуррентную нейронную сеть LSTM. На основе анализа их характеристик был сделан вывод о перспективности моделей машинного обучения для данной задачи.

    \item \textbf{Обоснован выбор и реализованы три модели}. Для каждой модели была разработана своя стратегия формирования признаков: для SARIMA использовались автокорреляции, для CatBoost — календарные, лаговые и статистические признаки, для LSTM — циклические признаки и признаки на основе скользящих окон.

    \item \textbf{Разработана и реализована методология MLOps}, включающая автоматизированный сбор данных, генерацию признаков и проведение экспериментов с использованием кросс-валидации для временных рядов (TimeSeriesSplit), что обеспечило корректность и воспроизводимость результатов.

    \item \textbf{Проведено экспериментальное исследование}, в ходе которого были получены следующие ключевые результаты:
    \begin{itemize}
        \item \textbf{Модель LSTM} показала наивысшее качество, достигнув MAPE = 0.89\%, что значительно превосходит техническое требование (MAPE < 10\%). Это подтверждает способность рекуррентных сетей улавливать сложные нелинейные зависимости.
        \item \textbf{Модель CatBoost} также удовлетворила требованиям с результатом MAPE = 8.90\%, представляя собой эффективный компромисс между качеством и сложностью реализации.
        \item \textbf{Модель SARIMA} не справилась с поставленной задачей (MAPE = 11.06\%), что демонстрирует ограниченную применимость классических линейных моделей к описанию сложных динамических процессов.
        \item Был выявлен и проанализирован эффект \textbf{утечки данных}, что подчеркнуло критическую важность правильной методологии валидации.
    \end{itemize}

    \item \textbf{Сформулированы практические рекомендации}. На основе анализа ошибок и сравнительных результатов модель LSTM рекомендована для внедрения в производственную среду в случаях, когда требуется максимальная точность прогноза. Модель CatBoost может использоваться как более простое и менее ресурсоемкое решение, также удовлетворяющее базовым требованиям.
\end{enumerate}

\hspace*{1.25cm}Таким образом, цель работы — разработка метода предиктивного анализа задержек — достигнута. Созданное решение позволяет с высокой точностью прогнозировать состояние видеоаналитического конвейера, что открывает возможности для превентивного реагирования на потенциальные сбои и повышения общей надежности системы. Практическая значимость работы заключается в создании готового к внедрению прототипа системы мониторинга, который может быть адаптирован для широкого круга систем реального времени.

\vspace{1cm}
\hspace*{1.25cm}\textbf{На защиту выносятся следующие положения:}
\begin{enumerate}
    \item \textbf{Комплексная методика предиктивного анализа}, включающая в себя специализированную инженерию признаков для данных мониторинга, гибридный подход к моделированию на основе декомпозиции рядов и строгую процедуру валидации на основе временной кросс-валидации.
    \item \textbf{Результаты экспериментального сравнения моделей} (SARIMA, CatBoost, LSTM) на реальных промышленных данных, которые доказывают практическую применимость и высокую точность (MAPE < 1\%) нейросетевого подхода для прогнозирования задержек в видеоаналитических системах и служат основанием для выбора оптимальной архитектуры в зависимости от требований к точности и ресурсам.
\end{enumerate}
