\chapter{Общие положения}

\section{Виды обледенения}

Существует два существенно различающихся вида обледенения при полете на больших высотах [5]. Первый вид – это обледенение в условиях крупных переохлажденных капель. Второй – в условиях кристаллов льда и смеси фаз. Обледенение в условиях жидких переохлажденных капель происходит в основном на входных элементах двигателя, а обледенение в условиях ледяных кристаллов и смеси фаз происходит в основном в проточной части компрессора, так как для прилипания необходимо их подплавление (\textit{Рисунок 2}). Рассмотрим первый вид поподробнее.

\begin{figure}[H]
	\centering
	\includegraphics[width=0.8\textwidth]{figures/chill-guy.jpeg}
	\caption*{Рисунок~2 – Виды обледенения при воздействии на двигатель различных условий атмосферного облака.}
	\label{fig:-0}
\end{figure}

\section{Обледенение переохлажденными каплями}

При температурах от 0\textdegree{}C до –30\textdegree{}C 
в атмосфере присутствуют переохлаждённые капли воды, остающиеся в жидком состоянии несмотря на отрицательные температуры. Это вызвано тем, что у них нет центров кристаллизации для замерзания. Поэтому, попадая на входные элементы двигателя, они мгновенно замерзают, образуя ледяные наросты, представленные на \textit{рисунке 3}. Этот процесс зависит от концентрации жидкой воды в набегающем потоке, распределения капель по размеру, аэродинамических условий и наличия или отсутствия обогрева поверхности, подвергающейся обледенению. Образовавшиеся ледяные наросты:
\begin{enumerate}[label=\arabic*)\hspace{1em}, leftmargin=2cm, itemsep=0em]
	\item Изменяют эффективный профиль лопатки и аэродинамику потока внутри двигателя;
	\item Снижают КПД двигателя;
	\item Повышают риск механического разрушения деталей, при отрыве больших ледяных наростов.
\end{enumerate}

Для защиты применяют противообледенительные системы, которые реализуют электрический обогрев или отбор горячего воздуха из камеры сгорания для обогрева поверхностей, подверженных обледенению, однако их эффективность ограничена интенсивностью облака и ресурсными возможностями двигателя.
\begin{figure}[H]
	\centering
	\includegraphics[width=0.8\textwidth]{figures/chill-guy.jpeg}
	\caption*{Рисунок~3 – Образование льда на элементах входа в двигатель.}
	\label{fig:-1}
\end{figure}

При обледенении жидкими переохлаждёнными каплями выделяют два типа: изморозь (\textit{рисунок 4а}) и глазурный лёд (\textit{рисунок 4б}) [5].

\begin{figure}[H]
	\centering
	\includegraphics[width=\textwidth]{figures/chill-guy.jpeg}
	\vspace{0.5em}
	\begin{tabular}{@{}p{0.48\textwidth}@{}p{0.48\textwidth}@{}}
		\centering а) & \centering б) \\
	\end{tabular}
	
	\caption*{Рисунок~4 – Схемы обледенения в условиях жидких переохлаждённых капель на аэродинамических поверхностях, а) изморозь и б) глазурный лёд.}
	\label{fig:-2}
\end{figure}

Изморозь, или rime ice, формируется, когда мелкие переохлаждённые капли мгновенно замерзают при ударе о поверхность. Такой лед выглядит молочно, имеет плотную структуру и относительно округлую форму.
Глазурный лёд, или glaze ice, возникает, когда более крупные капли не замерзают мгновенно при взаимодействии с поверхностью. Часть жидкости растекается по поверхности, образуя водяную пленку, которая затем замерзает, часто с образованием роговидных наростов. Эти процессы негативно влияют на аэродинамику и эффективность работы двигателя, поэтому важно понимать их механизмы для разработки эффективных мер защиты.

\section{Обледенение ледяными кристаллами и смесью фаз}

Второй вид обледенения возникает при попадании двигателя в атмосферные облака, содержащие кристаллы льда или смесь фаз. Ледяные кристаллы проникают в проточную часть компрессора двигателя и в процессе движения частично подплавляются, что обеспечивает возможность их прилипания к элементам проточной части компрессора. Интенсивное выпадение кристаллов на теплую поверхность приводит к ее постепенному охлаждению до нуля градусов Цельсия, после чего начинается процесс льдообразования. Механизмы формирования обледенения в условиях кристаллов значительно сложнее, чем в случае переохлажденных капель и зависят от концентрации и степени подплавления кристаллов, интенсивности обогрева поверхности, газодинамических характеристик потока, размера частиц, влажности и геометрии проточной части. Процесс льдообразования определяется не только термодинамическими процессами, но и эрозией поверхности льда.

Процесс обледенения в условиях ледяных кристаллов и смеси фаз сопровождается формированием конусообразных ледяных наростов, что приводит к существенно иным формам льда, показанным на \textit{рисунке 5}.

\begin{figure}[H]
	\centering
	\includegraphics[width=0.8\textwidth]{figures/chill-guy.jpeg}
	\caption*{Рисунок~5 – Рост льда на цилиндрической модели (красная линия – поверхность цилиндра) при различной степени подплавления кристаллов.}
	\label{fig:-3}
\end{figure}

При низком коэффициенте подплавления, т.е. отношению воды в жидкой фазе ко всей воде, кристаллы льда, попадая на поверхность, просто отскакивают, а при высоком – избыток жидкости смывает нарастающий лёд, оставляя поверхность недостаточно охлаждённой для нарастания льда. Исследования показывают, что существует оптимальный диапазон коэффициента подплавления примерно в пределах 5–25\%, где нарастание льда достигает пика (\textit{рисунок 6}) [5].

\begin{figure}[H]
	\centering
	\includegraphics[width=0.8\textwidth]{figures/chill-guy.jpeg}
	\caption*{Рисунок~6 – Гипотетическая степень обледенения в условиях ледяных кристаллов в зависимости от степени таяния, указывающая на период наибольшего нарастания в диапазоне 10-25\%.}
	\label{fig:-4}
\end{figure}

\section{Постановка задачи}

Для расчёта необходимо моделировать движение трёхфазной среды, включающей воздух, переохлаждённые капли и ледяные кристаллы. При этом учитываются аэродинамика потока, изменения в дисперсном составе частиц по мере движения, а также их взаимодействие с поверхностями двигателя.

Задача: Расчёт процессов обледенения двигателей

Для расчета обледенения необходимо знать начальные условия которые включают в себя:
\begin{enumerate}[label=\arabic*)\hspace{1em}, leftmargin=2cm, itemsep=0em]
	\item Аэродинамические поля воздушного потока внутри проточной части;
	\item Фрагментация капель и кристаллов в процессе их дробления при движении по проточной части компрессора;
\end{enumerate}
Задача данного исследования:
\begin{enumerate}[label=\arabic*)\hspace{1em}, leftmargin=2cm, itemsep=0em]
	\item Разработка модели распада капель как при движении в потоке, так и дроблении в процессе взаимодействия со стенкой;
	\item Разработка модели дробления кристаллов в процессе их механического взаимодействия с элементами проточной части компрессора.;
\end{enumerate} 