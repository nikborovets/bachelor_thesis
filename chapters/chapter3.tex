\chapter{Особенности программы}

В данной работе капли и кристаллы моделируются отдельно двумя программами. Хотя для них выполняются одни и те же уравнения движения, но их поведение при взаимодействии с поверхностью в корне отличается как видно из моделей описанных выше. Капля описывается как жидкая частица, её форма и размер меняются непрерывно под действием капиллярных и инерционных сил (числа Вебера и Рейнольдса) в процессе полета, поэтому в алгоритме учитывается постепенное приближение к «стабильному» диаметру.
Кристалл, напротив, рассматривается как твёрдое тело с фиксированным начальным диаметром, у которого при ударе в первую очередь включаются механизмы упругого и неупругого отскоков или фрагментации, основанного на безразмерном числе Видора и критериях прочности таких как модуль Юнга, энергия образования новой поверхности. Здесь не наблюдается плавного изменения диаметра в полёте, а изменение размеров происходит только при фрагментации в момент столкновения.

\section{Моделирование капель}

Рассмотрим ключевые особенности реализации программы моделирования движения и взаимодействия жидкой капли со стенкой в потоке воздуха. Программа написана на языке Python и использует библиотеки numpy, math и matplotlib. Ниже рассмотрены следующие аспекты:

\begin{enumerate}
	\item Структура кода и объектно-ориентированный подход
	\item Алгоритм расчёта динамики капли
	\item Обработка столкновения со стенкой
	\item Хранение данных и визуализация
\end{enumerate}

\noindent 1. Структура кода и объектно-ориентированный подход

Капля реализована в виде класса Drop, который хранит текущее положение $(x,y)$, скорости $(v_x,v_y)$, диаметр $d$, внешнюю силу $F_{ext}$ и параметры среды такие как плотность воздуха $\rho_a$, вязкость $\mu_a$ и скорость воздуха $v_a$. В этом классе реализованы функции для моделирования движения капли и ее взаимодействия с поверхностями.

\noindent 2. Алгоритм расчёта динамики капли

Сначала считается число Рейнольдса капли по формуле (2.9), затем по формуле (2.11) считается коэффициент сопротивления капли. Обновление скорости капли строится по дробной схеме в подходе Лагранжа, поэтому формула (2.8) упрощается до (2.13):
\begin{equation}
	\mathbf{v}_{i+1} = \frac{\mathbf{v}_i+\frac{3}{8}\frac{C_d \rho_a}{\rho_d r_d} v_a \triangle t +\mathbf{F}_{ext} \triangle t}{1+\frac{3}{8}\frac{C_d \rho_a}{\rho_d r_d} \triangle t},
\end{equation}
где $\triangle t$ --- шаг по времени, $\mathbf{v}_i$ --- вектор скорости на i-ом шаге, а $\mathbf{v}_{i+1}$ --- на i$+1$-ом.

\noindent 3. Обработка столкновения со стенкой

В данной реализации программы стенка задается вертикальной линией с $x=x_{wall}$, $y \in [y_{bottom},y_{top}]$, хотя стенку можно задавать любой. При моделировании на каждом шаге проверяется условие $x_i \geq x_{wall}$, если это условие выполняется, то капля столкнулась со стенкой и по нормальному числу Вебера (2.18) считается по модели Трухильо [4] вариант взаимодействия.

\noindent 4. Хранение данных и визуализация

 В классе Drop есть массив для хранения всех данных связанных с каплей и несущей фазой. В нем хранится 12 переменных $t_i, x_i, y_i, v_{x,i},v_{y,i},d_i,We_i,Re_i,d_{stab,i},\triangle v_i,v_{a,i}$, $We_{n,i}$, где $\triangle v_{i}$ --- модуль разности скоростей капли и воздуха. В качестве визуализации поведения капли строятся графики $d(t),x(t),We(t),Re(t),d_{stab}(t),\triangle v(t),v_a(t),We_n(t)$ для упрощения анализа поведения капли. В дополнение к этому производится анимация поведения капли на которой в динамике видно поведение капли.

\section{Моделирование кристаллов}

Ниже рассмотрены следующие особенности программы для расчета взаимодействия кристаллов:

\begin{enumerate}
	\item Объектно-ориентированная структура
	\item Число Видора и критерии взаимодействия со стенкой
	\item Сопротивление воздуха и обновление скоростей
	\item Хранение данных и визуализация
\end{enumerate}

\noindent 1. Объектно-ориентированная структура

Кристалл представлен классом Crystal, в котором хранится текущее положение $(x,y)$, скорость $(v_x,v_y)$, диаметр $d$, внешняя сила $F_{ext}$, температура кристалла и параметры прочности, такие как энергия активации и поверхностная энергия.

\noindent 2. Число Видора и критерии взаимодействия со стенкой

На каждом шаге для кристалла вычисляется безразмерное число Видора и проверяется так же как и для капли условие столкновения со стенкой. В зависимости от числа Видора происходит один из трех возможных случаев взаимодействия со стенкой согласно модели MUSIC-haic [3].

\noindent 3. Сопротивление воздуха и обновление скоростей

Для участков полета, где кристалл не взаимодействует со стенкой, считается число Рейнольдса $Re$ и коэффициент сопротивления $C_d$ так же, как и для капли. Обновление скорости происходит по той же формуле, что и для капли.

\noindent 4. Хранение данных и визуализация

Для кристаллов так же, как и для капель создан массив для хранения всей информации связанной с ним. В нем хранится 10 величин таких как $t_i,x_i,y_i,v_{x,i},v_{y,i},Re_i,\triangle v_i$, $v_{a,i},\mathcal{L}_i,d_i$. Для визуализации поведения кристалла строятся графики для $x(t),\mathcal{L}(t),Re(t)$, $\triangle v(t),v_a(t),v_x(t),v_y(t)$. Анимация траектории кристалла: красная вертикальная линия $x=x_{wall}$, которая обозначает стенку. Точка движется и отражается (или фрагментируется) в зависимости от условий. При фрагментации дополнительно выводится график распределения фрагментов по размерам (2.47). 