\chapter{Разработка моделей для оценки задержек}
\label{ch:modeling}

\hspace*{1.25cm}После всестороннего анализа данных на предыдущем этапе, текущая глава посвящена практической разработке и реализации моделей для оценки задержки в видеоаналитическом конвейере. Основное внимание уделяется двум ключевым этапам: формированию признаков, направленному на извлечение максимального количества полезной информации из временных рядов, и выбору, описанию и реализации моделей машинного обучения.

\section{Формирование признаков}
\label{sec:feature_engineering}

\hspace*{1.25cm}Формирование признаков --- это процесс преобразования исходных временных рядов в структурированный набор данных (таблицу), пригодный для обучения моделей машинного обучения. Качество и информативность признаков напрямую влияют на точность и обобщающую способность итоговой модели [6]. На основе анализа, проведенного в Главе 2, был сформирован следующий набор признаков.

\subsection{Календарные и временные признаки}

\hspace*{1.25cm}Эти признаки позволяют модели учитывать зависимости, связанные со временем суток, днем недели и общим течением времени. Они особенно полезны для моделей на основе деревьев решений, таких как CatBoost.

\begin{itemize}
	\item \textbf{Час дня (\texttt{hour})} и \textbf{день недели (\texttt{day\_of\_week})}: категориальные признаки, позволяющие модели улавливать суточные и недельные паттерны.
	\item \textbf{Признак выходного дня (\texttt{is\_weekend})}: бинарный флаг, принимающий значение 1, если день является субботой или воскресеньем, и 0 в противном случае.
	\item \textbf{Временной индекс (\texttt{time\_idx})}: монотонно возрастающая переменная, представляющая собой количество времени (в часах), прошедшее с начала обучающего периода. Этот признак помогает модели аппроксимировать долгосрочный тренд в данных.
\end{itemize}

\subsection{Циклические признаки}

\hspace*{1.25cm}Календарные признаки, такие как час дня или день недели, по своей природе цикличны (после 23:00 идет 00:00). Чтобы донести эту информацию до моделей, особенно нейронных сетей, используются тригонометрические преобразования.

\begin{equation}
	x_{sin} = \sin\left(\frac{2\pi x}{P}\right), \quad x_{cos} = \cos\left(\frac{2\pi x}{P}\right)
\end{equation}
где $x$ — исходное значение (например, час), а $P$ — период цикла (24 для часов, 7 для дней недели). Такой подход преобразует одну переменную в две, представляя ее на единичной окружности.

\subsection{Лаговые признаки (Lag features)}

\hspace*{1.25cm}Лаговые признаки — это значения временного ряда из прошлого, используемые в качестве предикторов для будущих значений. Они являются ключевым способом информирования модели об авторегрессионной структуре данных, выявленной при анализе ACF/PACF. Признак создается путем сдвига временного ряда на $k$ шагов назад:
\begin{equation}
	\text{lag}_k(t) = y(t-k)
\end{equation}
где $y(t-k)$ — значение целевой переменной в момент времени $t-k$.

\hspace*{1.25cm}В данном исследовании для CatBoost-модели использовались лаги: 1, 2, 4, 96, 192, 5760 шагов назад, что соответствует интервалам от 15 секунд до 24 часов. Такой выбор позволяет модели учитывать как непосредственную зависимость от предыдущих значений, так и суточную сезонность (лаг 5760 = 24 часа $\times$ 240 точек/час).

\subsection{Признаки на основе скользящего окна (Rolling-window features)}

\hspace*{1.25cm}Для захвата локальной динамики и структуры временного ряда вычисляются статистические показатели в пределах скользящего окна.

\begin{itemize}
	\item \textbf{Скользящее среднее (\texttt{rolling\_mean})}: сглаживает краткосрочные флуктуации и помогает выявить локальный тренд.
	\item \textbf{Скользящее стандартное отклонение (\texttt{rolling\_std})}: характеризует волатильность (изменчивость) ряда в недавнем прошлом.
\end{itemize}
Размер окна $w$ является гиперпараметром, который выбирается в зависимости от специфики модели и характера данных. В данном исследовании использовались различные наборы параметров для разных типов моделей:

\begin{itemize}
	\item \textbf{Для LSTM-модели}: окна размером 20 и 240 точек данных (соответствующие 5 минутам и 1 часу при 15-секундном интервале);
	\item \textbf{Для CatBoost-модели}: более широкий набор окон --- 4, 96, 192, 1920, 2880, 4320, 5760, 8640 точек данных (от 1 минуты до 36 часов), что позволяет модели улавливать как краткосрочные, так и долгосрочные паттерны.
\end{itemize}  % Формирование признаков
\section{Выбор и описание моделей}
\label{sec:model_selection}

На основе выводов, сделанных в Главе 2, для решения задачи прогнозирования были выбраны модели, представляющие два разных подхода: классическую статистику и современное машинное обучение.

\subsection{Модель SARIMA}
Сезонная авторегрессионная интегрированная скользящая средняя \\(SARIMA) — это статистическая модель, которая является расширением модели ARIMA и предназначена для работы с временными рядами, обладающими ярко выраженной сезонностью [7]. Выбор этой модели обоснован анализом ACF/PACF, который указал на наличие тренда, авторегрессионной зависимости и сезонных колебаний.

\subsection{Модель CatBoost}
CatBoost — это высокопроизводительная реализация градиентного бустинга над деревьями решений [8]. Она хорошо зарекомендовала себя в работе с разнородными табличными данными, эффективно обрабатывает категориальные признаки и не требует тщательной настройки гиперпараметров.

Особенностью предложенного в данной работе подхода является использование гибридной модели на основе CatBoost. Поскольку модели, основанные на деревьях решений, не способны экстраполировать тренд, была применена стратегия декомпозиции временного ряда. Исходный ряд был разделен на три компоненты: тренд, сезонность и остатки. Каждая из этих компонент прогнозировалась отдельно:
\begin{itemize}
    \item \textbf{Тренд} моделировался с помощью отдельной, более простой линейной модели.
    \item \textbf{Сезонная компонента и остатки} прогнозировались основной моделью CatBoost, которая эффективно работает со сложными нелинейными зависимостями после удаления тренда.
\end{itemize}
Итоговый прогноз получался путем суммирования прогнозов по каждой из компонент. Такой подход позволяет сочетать преимущества обоих типов моделей.

\subsection{Модель LSTM}
Сети с долгой краткосрочной памятью (Long Short-Term Memory, LSTM) — это разновидность рекуррентных нейронных сетей (RNN), специально разработанная для улавливания долгосрочных зависимостей в последовательных данных [9].Также архитектура LSTM позволяет эффективно бороться с проблемой затухающих градиентов.

В данной работе используется LSTM-архитектура со следующими характеристиками:
\begin{itemize}
	\item Входной слой принимает последовательности длиной \texttt{window\_size} временных шагов с количеством признаков, определяемым этапом формирования признаков;
	\item Один LSTM-слой с 64 нейронами;
	\item Полносвязный скрытый слой с 8 нейронами и функцией активации ReLU;
	\item Выходной слой с одним нейроном и линейной функцией активации для регрессии;
	\item Оптимизатор Adam с learning rate 0.0001, функция потерь --- Mean Squared Error.
\end{itemize}

\subsection{Дополнительные эксперименты с современными моделями}

В рамках исследования также проводились эксперименты с современными архитектурами для анализа временных рядов, такими как N-BEATS [20], а также с моделями из библиотек Time-Series-Library [10] и AutoTS [11] для оценки их применимости к данной задаче.

\subsubsection{Модели Time-Series-Library}

Были протестированы следующие модели на основе трансформеров и линейных архитектур:
\begin{itemize}
	\item \textbf{DLinear, PatchTST, iTransformer, Crossformer} --- успешно завершили обучение;
	\item \textbf{NLinear, Autoformer, FEDformer, Informer, TimesNet, Transformer} --- завершились с ошибками или превысили лимит времени выполнения.
\end{itemize}

\subsubsection{Модели AutoTS}

Библиотека AutoTS [11] предоставляет автоматизированный подход к выбору и настройке моделей временных рядов. Были протестированы модели:
\begin{itemize}
	\item \textbf{LastValueNaive} --- простая baseline модель;
	\item \textbf{SeasonalityMotif, SectionalMotif} --- модели на основе выявления паттернов;
	\item \textbf{GLS} --- обобщенный метод наименьших квадратов.
\end{itemize}

\subsubsection{Результаты дополнительных экспериментов}

Несмотря на современность указанных архитектур, результаты оказались неудовлетворительными по сравнению с основными моделями (SARIMA, CatBoost, LSTM). Это может быть обусловлено:
\begin{itemize}
	\item Недостаточной настройкой гиперпараметров для специфики данной задачи;
	\item Неоптимальным формированием признаков для трансформер-архитектур;
	\item Различиями в методологии предобработки данных между библиотеками.
\end{itemize}

В связи с этим для финальной оценки были выбраны три основные модели, показавшие наилучшее соотношение качества и стабильности результатов. % Выбор и описание моделей
\section{Метрики оценки качества}
\label{sec:evaluation_metrics}

Для оценки качества моделей прогнозирования используется набор метрик, позволяющих комплексно оценить точность оценки задержек. Выбор метрик обусловлен спецификой временных рядов и требованиями к практическому применению системы.

\subsection{Средняя абсолютная процентная ошибка (MAPE)}

MAPE является основной метрикой для оценки качества, поскольку обеспечивает интерпретируемость результатов в процентах:

\begin{equation}
	\text{MAPE} = \frac{100\%}{n} \sum_{i=1}^{n} \left| \frac{y_i - \hat{y}_i}{y_i} \right|
\end{equation}

где $y_i$ — истинное значение, $\hat{y}_i$ — прогнозируемое значение, $n$ — количество наблюдений. Согласно техническим требованиям, целевое значение MAPE должно быть менее 10\%.

\subsection{Среднеквадратичная ошибка (RMSE)}

RMSE чувствительна к выбросам и позволяет оценить общую точность модели:

\begin{equation}
	\text{RMSE} = \sqrt{\frac{1}{n} \sum_{i=1}^{n} (y_i - \hat{y}_i)^2}
\end{equation}

\subsection{Средняя абсолютная ошибка (MAE)}

MAE менее чувствительна к выбросам и показывает среднее отклонение прогнозов:

\begin{equation}
	\text{MAE} = \frac{1}{n} \sum_{i=1}^{n} |y_i - \hat{y}_i|
\end{equation}  % Метрики оценки качества
\section{Методология проведения экспериментов}
\label{sec:experiment_methodology}

Корректная оценка качества моделей временных рядов требует специального подхода к разделению данных, учитывающего временную структуру и предотвращающего утечку информации из будущего в прошлое.

\subsection{Кросс-валидация для временных рядов}

Для корректной оценки качества моделей применяется специализированная кросс-валидация временных рядов (TimeSeriesSplit), которая учитывает хронологический порядок данных и предотвращает утечку информации из будущего.

Метод TimeSeriesSplit работает следующим образом:
\begin{itemize}
	\item Данные разбиваются на $k$ фолдов, где каждый последующий фолд включает больше исторических данных для обучения;
	\item Для каждого фолда тестовая выборка всегда находится хронологически после обучающей;
	\item Внутри каждого фолда обучающие данные дополнительно разделяются на train и validation в пропорции, определяемой параметром \texttt{test\_size}.
\end{itemize}

\subsection{Процедура валидации}

Для каждого фолда кросс-валидации выполняется следующая последовательность действий:
\begin{enumerate}
	\item \textbf{Разделение данных}: фолд разбивается на train+validation и test согласно TimeSeriesSplit;
	\item \textbf{Внутреннее разделение}: train+validation дополнительно разделяется на обучающую и валидационную выборки;
	\item \textbf{Масштабирование}: параметры нормализации вычисляются только на обучающей выборке и применяются ко всем частям фолда;
	\item \textbf{Обучение модели}: модель обучается на train с валидацией на validation выборке;
	\item \textbf{Оценка качества}: финальная оценка производится на тестовой части фолда;
	\item \textbf{Сохранение результатов}: метрики каждого фолда сохраняются для последующего усреднения.
\end{enumerate}

Итоговые метрики качества вычисляются как среднее арифметическое соответствующих метрик по всем фолдам, что обеспечивает более надежную и несмещенную оценку производительности модели.

\subsection{Горизонт прогнозирования}

Все модели настраиваются для прогнозирования на 900 временных шагов вперед (3.75 часа), что соответствует практическим требованиям системы мониторинга для своевременного реагирования на потенциальные проблемы в видеоконвейере.  % Методология проведения экспериментов

