\chapter{Многофазная модель процесса}

\section{Математическая модель движения несущей фазы}

Для задания движения несущей фазы, то есть воздуха внутри двигателя, используются уравнения Навье-Стокса. Уравнение неразрывности для газа записывается в форме [7]: 
\begin{equation}
	\frac{\partial \rho}{\partial t}+\nabla(\rho \mathbf{V}_r)=0,
\end{equation}
где $\mathbf{V}_r$ --- относительная скорость газа во вращающейся системе координат. Уравнение импульса запишется в виде [7]:
\begin{equation}
	\frac{\partial \rho \mathbf{V}_r}{\partial t}+\nabla(\rho \mathbf{V}_r \mathbf{V}_r)= \nabla \sigma^{ij}_r+\rho \mathbf{g}+\mathbf{F}_\text{ext},
\end{equation}
где $\sigma^{ij}_r$ --- тензор напряжений, определяемый соотношением [7]:
\begin{equation}
	\sigma^{ij}_r=-\delta^{ij} p_a+\mu_a\left[\delta^{jk} \nabla_k v^i+\delta^{ik} \nabla_k v^j-\frac{2}{3}\delta^{ij} \nabla_k v^k\right] = -\delta^{ij} p_a+\tau^{ij},
\end{equation}
где $\tau^{ij}$ --- тензор касательных напряжений [7]:
\begin{equation}
	\tau^{ij}=\mu_a\left[\delta^{jk} \nabla_k v^i+\delta^{ik} \nabla_k v^j-\frac{2}{3}\delta^{ij} \nabla_k v^k\right],
\end{equation}
$p_a$ --- статическое давление воздуха, $\mu_a$ --- динамическая вязкость воздуха, определяемая по эмпирическому закону Сазерленда [7]:
\begin{equation}
	\frac{\mu_a}{\mu_\infty}=\left(\frac{T}{T_\infty}\right)^{\frac{3}{2}}\left(\frac{T_\infty+110}{T+110}\right),
\end{equation}
где $T$ --- статическая температура воздуха, $T_\infty = 288\,\text{К}$ и $\mu_\infty = 17{,}9 \cdot 10^{-6}\,\text{Па} \cdot \text{с}$. Вектор силы $\mathbf{F}_\text{ext}$ в уравнении (2.2) состоит из Кориолисовой и центробежной сил [7]:
\begin{equation}
	\mathbf{F}_\text{ext} = \mathbf{F}_\text{co}+\mathbf{F}_\text{ce}=-2 \rho(\vec{\Omega} \times \vec{V_r})-\rho \vec{\Omega} \times (\vec{\Omega} \times \vec{r}),
\end{equation}
где $\Omega$ --- угловая скорость вращения, а $r$ --- расстояние от оси вращения. 

\section{Математическая модель движения частиц в потоке}

Движение капель и кристаллов льда описывается уравнениями неразрывности и импульса в следующей форме [1]:
\begin{equation}
	\frac{\partial \alpha}{\partial t}+\nabla(\alpha \mathbf{V}_d)=0
\end{equation}
\begin{equation}
	\frac{\partial \alpha \mathbf{V}_d}{\partial t}+\nabla[\alpha \mathbf{V}_d \times \mathbf{V}_d]= \frac{C_d Re_d}{24K} (\mathbf{V}_a - \mathbf{V}_d)+\alpha\left(1-\frac{\rho_a}{\rho_d}\right)\frac{1}{Fr^2} + \mathbf{F}_{\text{ext}},
\end{equation}
где $\alpha$ --- концентрация частиц, где $\mathbf{V}_d$ --- вектор скорости частицы, $Re_d$ --- число Рейнольдса частицы, $K$ — инерционный параметр, $C_d$ --- коэффициент сопротивления частицы в потоке, $\mathbf{V}_a$ --- вектор скорости воздуха, $\mathbf{F}_{\text{ext}}$ --- вектор внешней силы. Число Рейнольдса рассчитывается следующим образом [4]:
\begin{equation}
	{Re_d} = \frac{\rho_a d \left| \mathbf{V}_a - \mathbf{V}_d \right|}{\mu_a},
\end{equation}
где $\rho_a$ --- плотность воздуха, $d$ --- диаметр частицы. Для инерционного параметра справедлива формула [6]:
\begin{equation}
	{K} = \frac{\rho_d d^2 V_{a,\infty}}{18 L_\infty \mu_a},
\end{equation}
где $\rho_d$ --- плотность частицы, $V_{a,\infty}$ --- характерная скорость воздуха, $L_\infty$ --- характерный размер задачи. В зависимости от числа Рейнольдса у частиц будет разный коэффициент сопротивления:
\begin{equation}
	C_d = 
	\begin{cases}
		\frac{24} {Re_d} (1 + 0.166 Re_d^{0.33}), & \text{если } Re_d < 350 \\
		0.178 Re_d^{0.217},  & \text{если } Re_d \geq 350
	\end{cases}
\end{equation}
Число Фруда, входящее в уравнение (2.8) рассчитывается следующим образом [7]:
\begin{equation}
	Fr = \frac{V_{a,\infty}}{\sqrt{L_\infty g_\infty}},
\end{equation}

При рассмотрении движения частиц в упрощенном случае, при подходе Лагранжа, в котором в отличие от Эйлерова подхода, все уравнения записываются для каждой частицы по отдельности, получим упрощенное уравнение импульса [1]: 
\begin{equation}
	\frac{d \mathbf{V}_d}{d t} = \frac{C_d Re_d}{24K} (\mathbf{V}_a - \mathbf{V}_d) + \mathbf{F}_{\text{ext}}
\end{equation}
В дальнейшем будет использоваться именно это уравнение для моделирования движения кристаллов и капель в потоке.

% Подключение модульных частей главы 2
\section{Модель разрушения капель в потоке}

Капли могут испытывать колебательные деформации, которые приводят к разрушению, вызванному ускорением, под действием аэродинамических сил. Изменение диаметра капли в процессе её разрушения определяется следующим дифференциальным уравнением в частных производных [1]:
\begin{equation}
	\frac{\mathrm{d} d}{\mathrm{d} t} = \frac{d_{stab} - d_0}{T},
\end{equation}
где $d_{stab}$ --- стабильный диаметр капли, $d_0$ --- начальный диаметр капли, $T$ --- безразмерное время разрыва капли. Для капель выделяют пять
различных механизмов разрушения, характеризующихся первоначальным состоянием капли.
Безразмерное время распада $T$ определяется следующим образом [1]:
\begin{equation}
	T = 
	\begin{cases}
		6 (We - 12)^{-0.25}, & 12 \leq We < 18 \\
		2.45(We-12)^{0.25}, & 18 \leq We < 45 \\
		14.1(We-12)^{-0.25}, & 45 \leq We < 350 \\
		0.766(We-12)^{0.25}, & 350 \leq We < 2700 \\
		5500, & We \geq 2700 
	\end{cases}
\end{equation}
Число Вебера, входящее в (2.15) определяется по формуле [1]:
\begin{equation}
	We = \frac{\rho_a \left| \mathbf{V}_a - \mathbf{V}_d \right|^2 d}{\sigma_d},
\end{equation}
где $\sigma_d$ --- коэффициент поверхностного натяжения капли. Максимальный устойчивый диаметр капли при прекращении всех процессов распада определяется с помощью критического числа Вебера [1]:
\begin{equation}
	We_{crit} = 12 \Rightarrow d_{stab} = \frac{12 \sigma_d}{\rho_a \left| \mathbf{V}_a - \mathbf{V}_d \right|^2},
\end{equation}

\section{Модель взаимодействия частиц со стенкой}

\subsection{Модель взаимодействия капель со стенкой}

При полете в канале капли и кристаллы могут взаимодействовать со стенками этого канала. Модель взаимодействия капли с твёрдой стенкой зависит от её скоростных и теплофизических параметров. У капель выделяют четыре различных режима представленных на \textit{рисунке 7} и основанных на безразмерном нормальном к стенке числе Вебера [4]:
\begin{equation}
	We_n = \frac{\rho_a \mathbf{V}_{d,n}^2 d}{\sigma_d},
\end{equation}
где $\mathbf{V}_{d,n}$ --- нормальная к поверхности скорость капли.

\begin{figure}[H]
	\centering
	
	\begin{subfigure}[t]{0.45\textwidth}
		\centering
		\includegraphics[width=\textwidth]{figures/chill-guy.jpeg}
		\caption{Прилипание}
		\label{fig:sub1}
	\end{subfigure}
	\hfill
	\begin{subfigure}[t]{0.45\textwidth}
		\centering
		\includegraphics[width=\textwidth]{figures/chill-guy.jpeg}
		\caption{Отскок}
		\label{fig:sub2}
	\end{subfigure}
	
	\vspace{1em}
	
	\begin{subfigure}[t]{0.45\textwidth}
		\centering
		\includegraphics[width=\textwidth]{figures/chill-guy.jpeg}
		\caption{Растекание}
		\label{fig:sub3}
	\end{subfigure}
	\hfill
	\begin{subfigure}[t]{0.45\textwidth}
		\centering
		\includegraphics[width=\textwidth]{figures/chill-guy.jpeg}
		\caption{Разбрызгивание}
		\label{fig:sub4}
	\end{subfigure}
	
	\caption*{Рисунок~7 - Варианты взаимодействия капли с твёрдой поверхностью}
	\label{fig:4grid}
\end{figure}

На основании различных диапазонов чисел Вебера определены основные режимы взаимодействия капли с поверхностью [4], они приведены в \textit{таблице 1}:

\begin{table}[H]
    \caption*{Таблица 1 - Режимы взаимодействия капли с поверхностью}
    \small
	\centering
	\begin{tabular}{|l|c|c|c|}
		\hline
		\makecell{\textbf{Режимы}\\\textbf{взаимодействия}} &
		\makecell{\textbf{Характерные}\\\textbf{числа Вебера}} &
		\makecell{\textbf{Результирующая}\\\textbf{скорость}} &
		\makecell{\textbf{Изменение диаметра и}\\\textbf{количества капель}} \\
		\hline
		\makecell{Прилипание капли \\\ к поверхности} & $We \leq 2$ & Скорость зануляется & Исключается из потока \\
		\hline
		Отскок капли & $2 < We \leq 10$ & Упругий отскок [4] & Не меняется \\
		\hline
		\makecell{Растекание капли \\\ по поверхности} & $10 < We \leq 1320La^{-0.183}$ & Скорость зануляется & Исключается из потока \\
		\hline
		\makecell{Реализуется \\\ разбрызгивание} & $We > 1320La^{-0.183}$ & \makecell{Вычисляется по \\\ модели Трухильо [4]} & \makecell{Количество и размер \\\ капель вычисляется \\\ по модели Трухильо [4]} \\
		\hline
	\end{tabular}
\end{table}
где число Лапласа вычисляется следующим образом:
\begin{equation}
	La = \frac{\sigma_d \rho_d d}{\mu_d^2},
\end{equation}
$\mu_d$ --- динамическая вязкость воды. Далее рассмотрим модель Трухильо для расчета изменения параметров взаимодействующих капель с поверхностью.

\subsection{Модель Трухильо}

Модель Трухильо [4] применяется для расчета изменения дисперсных характеристик потока, а именно изменения скоростей капель, их диаметров и количества при взаимодействии со стенкой. Рассмотрим как меняются параметры потока для всех четырех возможных случаев взаимодействия с поверхностью. Начнем с прилипания капли к поверхности.

\noindent Случай 1. Прилипание капли к поверхности

При прилипании капли к поверхности ее скорость зануляется, диаметр не меняется, а сама капля исключается из потока и не участвует в дальнейшем процессе льдообразования.

\noindent Случай 2. Отскок капли от поверхности и поступление во внешний поток

При отскоке капли от поверхности она поступает во внешний поток и ее параметры (указаны на \textit{рисунке 8}) меняются по модели Трухильо. Скорость меняется согласно соотношениям [1]:
\begin{equation}
	f_{u,t} = \frac{u_{t,s}}{u_{t,0}} = 0.85+0.0025\theta_0
\end{equation}
\begin{equation}
	f_{u,n} = \frac{u_{n,s}}{u_{n,0}} = -\left[0.993-0.0307\left(\frac{\pi}{2}-\theta_0\right)+0.0272\left(\frac{\pi}{2}-\theta_0\right)^2-0.0086\left(\frac{\pi}{2}-\theta_0\right)^3\right],
\end{equation}
\begin{equation*}
	\theta_0 = \arctan\left(\frac{u_{t,0}}{u_{n,0}}\right)
\end{equation*}
где $u_{t,s}$ --- тангенциальная составляющая отскочившей капли, $u_{t,0}$ --- тангенциальная составляющая падающей капли, $\theta_0$ --- угол между нормалью к поверхностью и вектором падающей капли, $u_{n,s}$ --- нормальная составляющая отскочившей капли, $u_{n,0}$ --- нормальная составляющая падающей капли. Размер капли при отскоке не меняется. 

\noindent Случай 3. Растекание капли по поверхости.

При растекании по поверхности капля исключается из потока и изменение ее параметров аналогично случаю прилипания.

\noindent Случай 4. Разбрызгивание капли.

Средняя скорость вторичных капель определяется с использованием эмпирической модели Трухильо следующим образом [1]:
\begin{equation}
	\frac{u_{t,s}}{u_{t,0}} = 0.85+0.0025\theta_0
\end{equation}
\begin{equation}
	\frac{u_{n,s}}{u_{n,0}} = -(0.12+0.002\theta_0)
\end{equation}

\begin{figure}[H]
	\centering
	\includegraphics[width=\textwidth]{figures/chill-guy.jpeg}
	\caption*{Рисунок~8 – Соответствующие параметры модели [1].}
	\label{fig:-6}
\end{figure}

При разбрызгивании из одной капли получается множество вторичных капель, количество которых определяется соотношением [4]:
\begin{equation}
	N = \frac{1}{22}\left\{0.0437\left[K\left(\frac{\left|u_d\right|}{u_{d,n}}\right)^2-K_{crit}\right]-44.92\right\}
\end{equation}  
где $K$ --- параметр разбрызгивания Ярина и Вейсса, который считается следующим образом [4]:
\begin{equation}
	K = \frac{3}{2}\left(\frac{LWC}{\rho_d}\right)^{-\frac{3}{8}}\left(Oh^{-\frac{2}{5}}We_n\right)^\frac{5}{16}
\end{equation}  
в который входит $LWC$ --- содержание жидкой воды в атмосфере [4], которое измеряется в граммах на метр кубический, $Oh$ --- критерий подобия Онезорге, характеризующий отношение сил вязкости к силам поверхностного натяжения и инерции [4]:
\begin{equation}
	Oh = \frac{\mu_a}{\sqrt{\rho_a \sigma_d d}}
\end{equation}

Параметр разбрызгивания увеличивается с увеличением диаметра, что делает разбрызгивание более вероятным для крупных переохлажденных капель, чем для обычных капель. При превышении величины параметра критического значения происходит разбрызгивание капли. Величина критического значения параметра Ярина и Вейсса [4]:
\begin{equation}
	K_{crit} = 17
\end{equation}

Учет массы капель, разбрызгивающихся при ударе о поверхность аэродинамического профиля, выполняется с применением эмпирической модели разбрызгивания Трухильо, которая использует понятие коэффициента потери массы $\varphi$. Уравнение было откалибровано для условий крупных переохдажденных капель. Количество массы разбрызганной воды получено из измерений Ярина и Вейсса и приведено в виде отношения объемов вторичных капель к падающим каплям [4]: 
\begin{equation}
	\varphi(K) = \frac{m_s}{m_0} = \frac{3.8}{\sqrt{K}}(1-\exp\{-0.85(K-K_{crit})\})
\end{equation}
где $m_s$ --- масса разбрызганной воды, $m_0$ --- масса падающей воды. Для $\varphi(K)$ зависимость от $K$ показана на \textit{рисунке 9} [4].
\begin{figure}[H]
	\centering
	\includegraphics[width=\textwidth]{figures/chill-guy.jpeg}
	\caption*{Рисунок~9 – Коэффициент потери массы.}
	\label{fig:-7}
\end{figure}
Зная количество вторичных капель и величину потери массы, средний диаметр вторичных капель может быть рассчитан по формуле [4]:
\begin{equation}
	d_s = \left(\frac{\varphi}{N}\right)^{\frac{1}{3}}d
\end{equation}

\subsection{Модель взаимодействия кристаллов со стенкой}

Модель взаимодействия кристалла с твёрдой поверхностью основана на использовании безразмерного числа Видора, которое показывает сооотношение между внутренней энергией кристалла при воздействии на поверхность и его поверхностной энергией [3]:
\begin{equation}
	\mathcal{L}=\frac{\rho_p d_p u^2_{p,n}}{12 e_{\sigma}},
\end{equation}
где $\rho_p$ и $d_p$ --- плотность и диаметр кристалла соответственно, $e_{\sigma}$ --- поверхностная энергия кристалла. Для неё справедлива формула:
\begin{equation}
	e_\sigma=e_{\sigma_0} \exp\left[\frac{Q_s}{R}\left(\frac{1}{T_p}-\frac{1}{T_0}\right)\right],
\end{equation}
где $e_{\sigma_0} =0{,}12\, \text{Дж}/\text{м}^2$ --- начальная поверхностная энергия кристалла, $Q_s = 48{,}2\, \text{кДж}/\text{моль}$ --- энергия активации, $R=8{,}31\,\text{Дж}/(\text{моль} \cdot \text{К})$ --- универсальная газовая постоянная, $T_p$ --- температура кристалла, а $T_0 = 253\, \text{К}$.

На основании различных диапазонов чисел Видора выделяют три основных режима взаимодействия кристалла с поверхностью [3], они приведены в \textit{таблице 2}:

\begin{table}[H]
	\caption*{Таблица 2 - Режимы взаимодействия кристалла с поверхностью}
	\small
	\centering
	\begin{tabular}{|l|c|c|c|}
		\hline
		\makecell{\textbf{Режимы}\\\textbf{взаимодействия}} &
		\makecell{\textbf{Характерные}\\\textbf{числа Видора}} &
		\makecell{\textbf{Результирующая}\\\textbf{скорость}} &
		\makecell{\textbf{Изменение диаметра и}\\\textbf{количества кристаллов}} \\
		\hline
		\makecell{Упругий отскок \\\ кристалла} & $\mathcal{L} \leq 0{,}5$ & \makecell{Меняется как для \\\ упругого отскока} & Не меняется \\
		\hline
		\makecell{Непругий отскок \\\ кристалла} & $0{,}5 < \mathcal{L} \leq 90$ & \makecell{Расчет по модели \\\ MUSIC-haic [3]} & Не меняется \\
		\hline
		\makecell{Фрагментация \\\ кристалла} & $\mathcal{L} > 90$ & \makecell{Расчет по модели \\\ MUSIC-haic [3]} & \makecell{Расчет по модели \\\ MUSIC-haic [3]} \\
		\hline
	\end{tabular}
\end{table}

\subsection{Модель MUSIC-haic}

Взаимодействие кристалла льда со стенкой представляет собой сложный процесс. Основными параметрами, влияющими на результаты и динамику, являются размер и скорость. Рассмотрим все три возможных варианта поподробнее.

\noindent Случай 1. Упругий отскок кристалла

При упругом отскоке кристалла от стенки его скорость меняется как для упругого оскока, то есть нормальная составляющая к поверхности меняет направление, не меняя величины, а тангенциальная составляющая сохраняется. Размер кристалла при этом не меняется.

\noindent Случай 2. Непругий отскок кристалла

Для неупругого отскока кристалла так же как и для упругого размер не меняется. Скорость изменяется по модели MUSIC-haic согласно соотношению [3]:
\begin{equation}
	\mathbf{v}_{p,s}=\xi_t (\mathbf{v}_p - \mathbf{v}_{p,n})-\xi_{nn} \mathbf{v}_{p,n},
\end{equation}
где $\mathbf{v}_{p,s}$ --- вектор скорости кристалла после отскока от поверхности, $\mathbf{v}_p$ --- вектор скорости перед ударом, $\mathbf{v}_{p,n}$ --- вектор нормальной к стенке скорости перед ударом, $\xi_{t} = 1$ --- коэффициент востановления тангенциальной компоненты скорости, $\xi_{nn}$ --- коэффициент востановления нормальной компоненты скорости, который вычисляется с помощью соотношения [3]:

\begin{equation}
	\xi_{nn} = \left(\frac{\mathcal{L}_1}{\mathcal{L}}\right)^{\frac{1}{3}},  \mathcal{L}_1 = 0.5
\end{equation}

\noindent Случай 3. Фрагментация кристалла 

Средняя скорость фрагментированных кристаллов определяется с помощью модели MUSIC-haic следующим образом [3]:

\begin{equation}
	\mathbf{v}_{p,s}=\xi_t (\mathbf{v}_p - \mathbf{v}_{p,n})-v_{p,n} (\xi_{nt} \mathbf{t}+\xi_{nn} \mathbf{n}),
\end{equation}
где $\mathbf{t}$ и $\mathbf{n}$ --- тангенциальный и нормальный к поверхности единичные вектора соответственно, $\xi_{nt}$ --- коэффициент восстановления, описывающий тангенциальную составляющую скорости, вызванную нормальной входящей составляющей скорости. Он вычисляется по формуле:

\begin{equation}
	\xi_{nt} = k \frac{1}{v_{p,n}} \int\limits_{r_{p,min}}^{r_{p,max}} \frac{\gamma-1}{r_{p,max}^{\gamma-1}-r_{p,min}^{\gamma-1}} r_f^{-\gamma} \int\limits_{0}^{r_p-r_f} \int\limits_{0}^{2 \pi} \int\limits_{0}^{\pi} \frac{1}{\frac{4}{3}(r_p-r_f)^3} k \dot{\varepsilon} (r_p+r\cos \theta) r^2 \sin \theta dr d\theta d\phi dr_f
\end{equation}
после взятия интеграла получается:
\begin{equation}
	\xi_{nt} = k \frac{3}{32}\pi v_{p,n}^{-\frac{1}{2}} \rho_p^{-\frac{1}{4}} Y_c^{\frac{1}{4}}\left[1-\frac{1}{r_p}\frac{r_{p,max}^{2-\gamma}-r_{p,min}^{2-\gamma}}{r_{p,max}^{1-\gamma}-r_{p,min}^{1-\gamma}}\right],
\end{equation}
где параметр $k \approx 0{,}909$ используется для приведения модели в соответствие с имеющимися экспериментальными данными, $\rho_p$ --- плотность льда, $r_p$ --- радиус налетающего кристалла, а $r_{p,max}$ и $r_{p,min}$ --- радиусы самого маленького и самого большого фрагментированного кристалла соответственно, $\gamma = 2{,}73$. Что касается предела текучести при сжатии $Y_c$, то экспериментальные данные
свидетельствуют о том, что он зависит от скорости деформации и от температуры. В качестве зависимости от скорости деформации используется полиномиальная подгонка. Температурная зависимость
квазистатического предела текучести льда при сжатии описывается
экспериментальными данными для которых простая
линейная подгонка дает удовлетворительную точность. Полученная формула гласит [8]:
\begin{equation}
	Y_c(T,\dot{\varepsilon})=Y_{c,0} f_{y_c}(\dot{\varepsilon}) g_{y_c}(T),
\end{equation}
где контрольное значение квазистатического предела текучести при сжатии $Y_{c,0} = 5{,}2 \cdot 10^{6}\,\text{Па}$, а для $f_{y_c}(\dot{\varepsilon})$ и $g_{y_c}(T)$ справедливы соотношения [8]:
\begin{equation}
	\begin{split}
		f_{y_c}(\dot{\varepsilon}) = \exp\bigg(
		&3.596 \cdot 10^{-4} \ln^3 \left(\frac{\dot{\varepsilon}}{\dot{\varepsilon}_0}\right) 
		-7.47 \cdot 10^{-3} \ln^2 \left(\frac{\dot{\varepsilon}}{\dot{\varepsilon}_0}\right) \\
		&+1.118 \cdot 10^{-1} \ln \left(\frac{\dot{\varepsilon}}{\dot{\varepsilon}_0}\right)
		+1.062 \bigg)
	\end{split}
\end{equation}
\begin{equation}
	g_{y_c}(T)=-28.7\left(\frac{T}{T_{c,0}}\right)+30.21,
\end{equation}
где $\dot{\varepsilon}$ --- скорость деформации кристалла, а $\dot{\varepsilon}_0$ --- параметр нормализации скорости деформации, $T_{c,0}=273\, \text{К}$. Оценка скорости деформации [8]: 
\begin{equation}
	\dot{\varepsilon} \sim v_{n,0}^{\frac{1}{2}} d_p^{-1} \rho_p^{-\frac{1}{4}} Y_c^{\frac{1}{4}}
\end{equation}

Как видно из уравнения (2.40) для того чтобы посчитать $\dot{\varepsilon}$ необходимо знать $Y_c$, но для $Y_c$ в свою очередь нужно знание $\dot{\varepsilon}$. Проблема решается просто: сначала для расчета $\dot{\varepsilon}$ по формуле (2.40) вместо $Y_c$ подставляется $Y_{c,0}$, а затем уже по формуле (2.37) вычисляется $Y_c$. Таким образом происходит итеративная процедура до тех пор пока не достигнется желаемая точность. Для $\dot{\varepsilon}_0$ справедливо соотношение [8]:
\begin{equation}
	\dot{\varepsilon}_0 = \frac{1}{6} P_{s,0}^3 c^{-3} \rho_p^{-2} G_c^{-1},
\end{equation}
где $P_{s,0} = 10^6\, \text{Па}$ --- постоянное исходное растягивающее давление, $c$ --- скорость звука и $G_c$ --- энергия на единицу площади, необходимая для создания новой поверхности разлома. Скорость звука в кристалле льда вычисляется как:
\begin{equation}
	c = \sqrt{\frac{E}{\rho_p}},
\end{equation}
где $E = 10{,}5 \cdot 10^9\,\text{Па}$ --- модуль Юнга.
\begin{equation}
	G_c = G_{c,0} \exp \left(\frac{Q_s}{R T}-\frac{Q_s}{R T_0}\right),
\end{equation}
где $G_{c,0} = 0{,}12\,\text{Дж}/\text{м}^2$.

Рассмотрим дальше какие получаются диаметры фрагментированных кристаллов. Для этого сначала считается максимальный диаметр фрагментированного кристалла [3]:
\begin{equation}
	d_{p,max}=C_f s_0 \left(\frac{\dot{\varepsilon}}{\dot{\varepsilon}_0}\right)^\alpha,
\end{equation}
где $C_f= 42{,}824$, $\alpha=-0{,}885$ [8], $s_0$ --- параметр нормализации для размера фрагмента, который вычисляется согласно соотношению [8]:
\begin{equation}
	s_0=12 E G_c P_{s,0}^{-2}
\end{equation}
Затем считается минимальный диаметр фрагментированного кристалла:
\begin{equation}
	d_{p,min}=0.015 d_{p,max}
\end{equation}
Теперь зная размер самого маленького и самого больших фрагментов льда строится распределение кристаллов по размерам [9]:
\begin{equation}
	Q(s)=1-\exp\left[-\left(\frac{s}{s_c}\right)^{k_s}\right],
\end{equation}
где $s_c=0{,}75$, $k_s=5{,}8$, $s = d/d_{p,max}$. $Q(s)$ рассчитывается путем суммирования массы
всех фрагментов, размеры которых меньше заданного размера $s$. Размер
фрагмента оценивается как диаметр сферы с одинаковой
массой, значения нормируются на диаметр образца. На \textit{рисунке 10} [9] можно видеть экспериментальные распределения по размерам, полученные для разных скоростей соударения кристаллов со стенкой. Зеленой пунктирной линией показана аппроксимация формулой (2.47). При увеличении скорости соударения кривая распределения смещается влево, то есть уменьшается максимальный диаметр получившихся осколков.

\begin{figure}[H]
	\centering
	\includegraphics[width=\textwidth]{figures/chill-guy.jpeg}
	\caption*{Рисунок~10 – Распределение фрагментов по размерам.}
	\label{fig:-8}
\end{figure} 
