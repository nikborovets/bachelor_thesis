% Основные пакеты
\usepackage{tabularx}
\usepackage{makecell}
\usepackage{subcaption}
\usepackage{array}
\usepackage{graphicx}
\usepackage{float}
\usepackage{caption}
\usepackage{amsmath, amssymb, enumitem, calc, indentfirst, amsthm}
\usepackage[normalem]{ulem}
\usepackage{fancyhdr}                               % Для настройки нумерации страниц согласно ГОСТ
\usepackage[hidelinks]{hyperref} % Убираем рамки вокруг ссылок
\usepackage{pdflscape} % Для альбомной ориентации страниц в PDF
\usepackage{etoolbox}
\usepackage{ragged2e}
\justifying

% Установка абзацного отступа 1.25 см
\setlength{\parindent}{1.25cm}

\captionsetup[figure]{labelsep=endash,justification=centering,singlelinecheck=false,font=normalfont,name=Рисунок}
\captionsetup[table]{labelsep=endash,justification=raggedright,singlelinecheck=false,font=normalfont,name=Таблица}

% Настройка нумерации страниц согласно п.2.5 ГОСТ
% Арабскими цифрами, сквозная нумерация, номер в центре нижней части листа
\pagestyle{fancy}
\fancyhf{}                                          % Очищаем все колонтитулы
\fancyfoot[C]{\thepage}                            % Номер страницы по центру внизу
\renewcommand{\headrulewidth}{0pt}                 % Убираем линию сверху
\renewcommand{\footrulewidth}{0pt}                 % Убираем линию снизу 