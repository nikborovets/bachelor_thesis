% Настройки оглавления в российском стиле (tocloft)
\usepackage{tocloft}

% Заголовок оглавления
\renewcommand{\contentsname}{\centering СОДЕРЖАНИЕ}

% Отступы для разных уровней
\setlength{\cftchapindent}{0pt}      % Главы без отступа
\setlength{\cftsecindent}{0pt}     % Разделы с отступом 0pt (был 1.5em)  
\setlength{\cftsubsecindent}{0pt}    % Подразделы с отступом 0pt (был 3em)

% Ширина номеров
\setlength{\cftchapnumwidth}{2em}    % Ширина для номеров глав
\setlength{\cftsecnumwidth}{3em}   % Ширина для номеров разделов
\setlength{\cftsubsecnumwidth}{4em}  % Ширина для номеров подразделов

% Отточия (точки между названием и номером страницы)
\renewcommand{\cftchapleader}{\cftdotfill{\cftdotsep}}    % Отточия для глав
\renewcommand{\cftsecleader}{\cftdotfill{\cftdotsep}}     % Отточия для разделов
\renewcommand{\cftsubsecleader}{\cftdotfill{\cftdotsep}}  % Отточия для подразделов

% Шрифты (убираем жирность с номеров глав)
\renewcommand{\cftchapfont}{\normalfont}      % Обычный шрифт для названий глав
\renewcommand{\cftchappagefont}{\normalfont}  % Обычный шрифт для номеров страниц глав
\renewcommand{\cftsecfont}{\normalfont}       % Обычный шрифт для разделов
\renewcommand{\cftsubsecfont}{\normalfont}    % Обычный шрифт для подразделов

% Интервалы между строками оглавления
\setlength{\cftbeforechapskip}{0pt}    % Без дополнительных отступов между главами
\setlength{\cftbeforesecskip}{0pt}     % Без дополнительных отступов между разделами
\setlength{\cftbeforesubsecskip}{0pt}  % Без дополнительных отступов между подразделами 